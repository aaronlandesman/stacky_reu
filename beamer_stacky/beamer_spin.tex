%%%%%%%%%%%%%%%%%%%%%
%   AMS packages    %
%%%%%%%%%%%%%%%%%%%%%
\documentclass{beamer}
\usepackage{amsmath}
\usepackage{amsxtra}
\usepackage{amscd}
\usepackage{amsthm}
\usepackage{amsfonts}
\usepackage{amssymb}
\usepackage{eucal}
%\usepackage[all]{xy}
\usepackage{graphicx}
\usepackage{comment}
\usepackage{amssymb}
\usepackage{mathrsfs}
\usepackage{latexsym,amsmath,amscd,amssymb,epsfig,verbatim}
\usepackage{tikz}
\usetikzlibrary{arrows,shapes}
\usepackage{tikz-qtree}
\tikzset{level distance=50pt,
    sibling distance=7pt,
    every tree node/.style={align=center},}

\newtheorem{thm}{Theorem}
\newtheorem{cor}[thm]{Corollary}
\newtheorem{lem}[thm]{Lemma}
\newtheorem{prop}[thm]{Proposition}
\newtheorem{ex}[thm]{Exercise}
\newtheorem{conjecture}{Conjecture}
%\newtheorem*{conjecture*}{Conjecture}

\theoremstyle{remark}
\newtheorem{rem}[thm]{Remark}
\newtheorem{eg}[thm]{Example}

%\newtheorem{counterexample}[thm]{Counterexample}
\newtheorem{defn}[thm]{Definition}
%\newtheorem{claim}[thm]{Claim}
%\newtheorem{note}[thm]{Notation}
%\newtheorem{warning}[thm]{Warning}
%\newtheorem{variant}[thm]{Variant}
%\newtheorem{question}[thm]{Question}
%\newtheorem{construction}[thm]{Construction}
%\newtheorem{terminology}[thm]{Terminology}
%\newtheorem{convention}[thm]{Convention}

%%%%%%%%%%%%%%%%%%%%%%%%% custom commands %%%%%%%%%%%%%%%%%%%%%%%%%%%

\newcommand\nc{\newcommand}
\nc\on{\operatorname}
\nc\renc{\renewcommand}
\newcommand\ssec{\subsection}
\newcommand\sssec{\subsubsection}
\newcommand\BH{{\mathbb H}}
\newcommand\bO{{\mathbf O}}
\newcommand\CC{{\mathcal C}}
\newcommand\BN{{\mathbb N}}
\newcommand\BC{{\mathbb C}}
\newcommand\BF{{\mathbb F}}
\newcommand\BR{{\mathbb R}}
\newcommand\BQ{{\mathbb Q}}
\newcommand\BP{{\mathbb P}}
\newcommand\BBZ{{\mathbb Z}}
\newcommand\uR{\underline{R}}
\newcommand\uZ{\underline{\BBZ}}
\newcommand\CF{{\mathcal F}}
\newcommand\uCF{\underline{{\mathcal F}}}
\newcommand\BZ{{\mathbb Z}}
\newcommand\BA{{\mathbb A}}
\newcommand\fa{{\mathfrak a}}
\newcommand\fp{{\mathfrak p}}
\newcommand\fq{{\mathfrak q}}
\newcommand\fm{{\mathfrak m}}
\newcommand\so{{\mathscr O}}
\newcommand\sg{{\mathscr G}}
\newcommand \sx{{\mathscr X}}

\newcommand\scm{{\mathscr M}}
\newcommand\scn{{\mathscr N}}
\newcommand\scf{{\mathscr F}}
\newcommand\scg{{\mathscr G}}
\newcommand\sco{{\mathscr O}}
\newcommand\sch{{\mathscr H}}
\newcommand\scl{{\mathscr L}}
\newcommand\sci{{\mathscr I}}

\newcommand{\id}{\mathrm{id}}
\newcommand\im{\text{im }}
\newcommand\coker{\text{coker}}
\newcommand \spec{\text{Spec }}
\newcommand \proj{\text{Proj }}
\newcommand \rspec{\textit{Spec }}
\newcommand \rproj{\textit{Proj }}
\newcommand{\gal}{\mathrm{Gal}}

\newcommand \trdeg{\text{tr. deg }}
\newcommand \codim{\text{codim}}
\newcommand \rk{\text{rk }}
\DeclareMathOperator\di{Div}
\newcommand \depth{\text{depth }}
\DeclareMathOperator{\ord}{ord}
\DeclareMathOperator{\sym}{Sym}

%%%%%%%%%%%%%%%%%%%%% cring custom commands %%%%%%%%%%%%%%%%%%%%%%%%%

\newcommand \subhalf[1]{\frac{{#1} - 1}{2{#1}}}
\newcommand{\se}[1]{\section*{Problem #1}}
\newcommand{\halfcan}{L}
\DeclareMathOperator{\Supp}{Supp}
\DeclareMathOperator{\initial}{in}
\DeclareMathOperator{\gin}{gin}
\DeclareMathOperator{\Eff}{Eff}
\DeclareMathOperator{\sat}{sat}


\newcommand\Wider[2][3em]{%
\makebox[\linewidth][c]{%
  \begin{minipage}{\dimexpr\textwidth+#1\relax}
  \raggedright#2
  \end{minipage}%
  }%
}


\AtBeginSection[]{
	\begin{frame}{Outline of Talk}
		\tableofcontents[currentsection]
	\end{frame}
}

%\usepackage{beamerthemeshadow}

\mode<presentation>{}
\usetheme{CambridgeUS}
\usecolortheme{beaver}

%%%%%%%%%%%%   Title slide info  %%%%%%%%%%%%%
\title{Log Spin Canonical Rings}
\author{Aaron Landesman\inst{1}, Peter Ruhm\inst{2}, Robin Zhang\inst{3}}
 
\institute[] % (optional)
{
  \inst{1}
	Harvard University
	,
  \inst{2}
	Stanford University
  ,
  \inst{3}
	Stanford University
}

\begin{document}

%%%%%%%%%%%%  Title Frame  %%%%%%%%%%%%%%%%
\begin{frame}
	\titlepage
\end{frame}

%%%%%%%%%%%%%%%%%%%%   Background Section   %%%%%%%%%%%%%%%%%%%
\section{Background} 

\begin{frame}
\frametitle{Background: Introduction}
Let $R$ be.

\pause
\begin{exampleblock}{Example of blah}
Example
\end{exampleblock}

\pause
\[
   M(\Gamma)
\]

\pause
\begin{alertblock}{Log Spin Canonical Ring Definition}
The \textbf{log spin canonical ring} of a stacky curve is
\[
	R(\sx, \Delta, \halfcan) = \bigoplus_{d \geq 0} H^0(\sx, \lfloor d \halfcan \rfloor)
\]
\end{alertblock}

\end{frame}


\begin{frame}
\frametitle{Background: Stacky Curves}
Let $R$ be.

\pause
\begin{exampleblock}{Example of blah}
Example
\end{exampleblock}

\pause
\[
   M(\Gamma)
\]

\pause
\begin{alertblock}{Log Spin Canonical Ring Definition}
The \textbf{log spin canonical ring} of a stacky curve is
\[
	R(\sx, \Delta, \halfcan) = \bigoplus_{d \geq 0} H^0(\sx, \lfloor d \halfcan \rfloor)
\]
\end{alertblock}

\end{frame}


\begin{frame}
\frametitle{Background: Monomial Ordering}
Let $R$ be.

\pause
\begin{exampleblock}{Example of blah}
Example
\end{exampleblock}

\pause
\[
   M(\Gamma)
\]

\pause
\begin{alertblock}{Log Spin Canonical Ring Definition}
The \textbf{log spin canonical ring} of a stacky curve is
\[
	R(\sx, \Delta, \halfcan) = \bigoplus_{d \geq 0} H^0(\sx, \lfloor d \halfcan \rfloor)
\]
\end{alertblock}

\end{frame}


\begin{frame}
\frametitle{Previous Work}
Let $R$ be.

\pause
\begin{exampleblock}{Example of blah}
Example
\end{exampleblock}

\pause
\[
   M(\Gamma)
\]

\pause
\begin{alertblock}{Log Spin Canonical Ring Definition}
The \textbf{log spin canonical ring} of a stacky curve is
\[
	R(\sx, \Delta, \halfcan) = \bigoplus_{d \geq 0} H^0(\sx, \lfloor d \halfcan \rfloor)
\]
\end{alertblock}

\end{frame}

%%%%%%%%%%%%%%%%%%%%   Main Result Section   %%%%%%%%%%%%%%%%%%%
\section{Main Result}

\begin{frame}
\frametitle{Main Result}
Our main result gives the following bound

\pause
\begin{thm}
\label{thm:main}
Let $(\sx, \Delta, \halfcan)$ be a log spin curve over a perfect field $k$, so
that $\sx$ has signature $\sigma = (0; e_1, \ldots, e_r; \delta)$. Then the
log spin canonical ring

\begin{align*}
	R := R(\sx, \Delta, \halfcan) = \bigoplus_{d \geq 0} H^0(\sx, \lfloor d \halfcan \rfloor)
\end{align*}

\noindent
is generated as a $k$-algebra by elements of degree at most $e =
\max(5, e_1, \ldots, e_r)$. Furthermore, $R$ has relations generated
in degrees up to
$\begin{cases}
	\max(11, 2e_1, \ldots, 2e_r) & g \geq 2 \\
	\max(10, 2e_1, \ldots, 2e_r) & g < 2
\end{cases}$

\noindent
so long as $\sigma$ is not in a finite list of exceptional
cases.
\end{thm}

\end{frame} 

\begin{frame}
\frametitle{Main Idea}
Let $R$ be.

\pause
\begin{exampleblock}{Example of blah}
Example
\end{exampleblock}

\pause
\[
   M(\Gamma)
\]

\pause
\begin{alertblock}{Log Spin Canonical Ring Definition}
The \textbf{log spin canonical ring} of a stacky curve is
\[
	R(\sx, \Delta, \halfcan) = \bigoplus_{d \geq 0} H^0(\sx, \lfloor d \halfcan \rfloor)
\]
\end{alertblock}

\end{frame}


\begin{frame}
\frametitle{Adding Points}
Let $R$ be.

\pause
\begin{exampleblock}{Example of blah}
Example
\end{exampleblock}

\pause
\[
   M(\Gamma)
\]

\pause
\begin{alertblock}{Log Spin Canonical Ring Definition}
The \textbf{log spin canonical ring} of a stacky curve is
\[
	R(\sx, \Delta, \halfcan) = \bigoplus_{d \geq 0} H^0(\sx, \lfloor d \halfcan \rfloor)
\]
\end{alertblock}

\end{frame}

\begin{frame}
\frametitle{Raising Stabilizer Order}
Let $R$ be.

\pause
\begin{exampleblock}{Example of blah}
Example
\end{exampleblock}

\pause
\[
   M(\Gamma)
\]

\pause
\begin{alertblock}{Log Spin Canonical Ring Definition}
The \textbf{log spin canonical ring} of a stacky curve is
\[
	R(\sx, \Delta, \halfcan) = \bigoplus_{d \geq 0} H^0(\sx, \lfloor d \halfcan \rfloor)
\]
\end{alertblock}

\end{frame}

%%%%%%%%%%%%%%%%%%%%   Inductive Section   %%%%%%%%%%%%%%%%%%%
\section{Inductive Lemmas} 

\begin{frame}
\frametitle{Inductive Lemmas}
Let $R$ be.

\pause
\begin{exampleblock}{Example of blah}
Example
\end{exampleblock}

\pause
\[
   M(\Gamma)
\]

\pause
\begin{alertblock}{Log Spin Canonical Ring Definition}
The \textbf{log spin canonical ring} of a stacky curve is
\[
	R(\sx, \Delta, \halfcan) = \bigoplus_{d \geq 0} H^0(\sx, \lfloor d \halfcan \rfloor)
\]
\end{alertblock}

\end{frame}

%%%%%%%%%%%%%%%%%%%%   High Genus Section   %%%%%%%%%%%%%%%%%%%
\section{High Genus} 

\begin{frame}
\frametitle{High Genus}
Let $R$ be.

\pause
\begin{exampleblock}{Example of blah}
Example
\end{exampleblock}

\pause
\[
   M(\Gamma)
\]

\pause
\begin{alertblock}{Log Spin Canonical Ring Definition}
The \textbf{log spin canonical ring} of a stacky curve is
\[
	R(\sx, \Delta, \halfcan) = \bigoplus_{d \geq 0} H^0(\sx, \lfloor d \halfcan \rfloor)
\]
\end{alertblock}

\end{frame}

%%%%%%%%%%%%%%%%%%%%   Genus One Section   %%%%%%%%%%%%%%%%%%%
\section{Genus One} 

\begin{frame}
\frametitle{Genus One}
Let $R$ be.

\pause
\begin{exampleblock}{Example of blah}
Example
\end{exampleblock}

\pause
\[
   M(\Gamma)
\]

\pause
\begin{alertblock}{Log Spin Canonical Ring Definition}
The \textbf{log spin canonical ring} of a stacky curve is
\[
	R(\sx, \Delta, \halfcan) = \bigoplus_{d \geq 0} H^0(\sx, \lfloor d \halfcan \rfloor)
\]
\end{alertblock}

\end{frame}

%%%%%%%%%%%%%%%%%%%%   Genus Zero Section   %%%%%%%%%%%%%%%%%%%
\section{Genus Zero} 

\begin{frame}
\frametitle{Genus Zero}
Let $R$ be.

\pause
\begin{exampleblock}{Example of blah}
Example
\end{exampleblock}

\pause
\[
   M(\Gamma)
\]

\pause
\begin{alertblock}{Log Spin Canonical Ring Definition}
The \textbf{log spin canonical ring} of a stacky curve is
\[
	R(\sx, \Delta, \halfcan) = \bigoplus_{d \geq 0} H^0(\sx, \lfloor d \halfcan \rfloor)
\]
\end{alertblock}

\end{frame}

\begin{frame}
\frametitle{Genus Zero Cases}
Let $R$ be.

\pause
\begin{exampleblock}{Example of blah}
Example
\end{exampleblock}

\pause
\[
   M(\Gamma)
\]

\pause
\begin{alertblock}{Log Spin Canonical Ring Definition}
The \textbf{log spin canonical ring} of a stacky curve is
\[
	R(\sx, \Delta, \halfcan) = \bigoplus_{d \geq 0} H^0(\sx, \lfloor d \halfcan \rfloor)
\]
\end{alertblock}

\end{frame}

%%%%%%%%%%%%%%%%%%%%   Further Research Section   %%%%%%%%%%%%%%%%%%%
\section{Further Research} 

\begin{frame}
\frametitle{Further Research}
Let $R$ be.

\pause
\begin{exampleblock}{Example of blah}
Example
\end{exampleblock}

\pause
\[
   M(\Gamma)
\]

\pause
\begin{alertblock}{Log Spin Canonical Ring Definition}
The \textbf{log spin canonical ring} of a stacky curve is
\[
	R(\sx, \Delta, \halfcan) = \bigoplus_{d \geq 0} H^0(\sx, \lfloor d \halfcan \rfloor)
\]
\end{alertblock}

\end{frame}


\begin{frame}
\frametitle{Acknowledgments}
Ken Ono!
David Zureick-Brown!
Great math atmosphere!

\end{frame}

\end{document}





