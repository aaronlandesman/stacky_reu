\begin{frame}{Ring of Modular forms}






\begin{definition}[Ring of Modular forms]
\[
M(\Gamma) := \bigoplus_{k \in 2\Z_{\geq 0}} M_k(\Gamma)
\]

\end{definition}


\begin{example}
\[
M(\SL_2(\Z)) \cong \C[E_4,E_6]
\]
\end{example}


\begin{theorem}[Wagreich]
$M(\Gamma)$ is generated by two elements if and only if 
\[
\Gamma = \SL_2(\Z), \Gamma_0(2), \text{or } \Gamma(2).
\]

\end{theorem}




%   \begin{definition}
%     \begin{enumerate}
%       \itemsep1em
%     \item $\Gamma$ Fuchsian group (e.g.~$\Gamma = \Gamma_0(N) \subset
%       \SL_2(\Z)$).
%     \item $\Gamma \circlearrowleft \calH$.
%     \item \textbf{Modular form}: $f \colon \calH \to \C$ such that 
% \[

% \]
%     \item $\bigoplus M_k(\Gamma,R)$ ring of modular forms.
%     \end{enumerate}
%   \end{definition}



\end{frame}
%%% Local Variables: 
%%% mode: latex
%%% TeX-master: "../ZBVoight-UNC-canonical-stacky"
%%% End: 
