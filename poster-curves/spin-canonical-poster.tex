\documentclass[landscape, a0, largefonts]{sciposter}
\usepackage{tcolorbox}
\usepackage{amsmath}
\usepackage{amssymb}
\usepackage{bm}
\usepackage{multicol}
\usepackage{graphicx}
\usepackage[english]{babel}
\usepackage{color}
\usepackage{wrapfig}
\usepackage{caption}
\usepackage{booktabs}
\usepackage{array}
\usepackage{ytableau}
\usepackage{verbatim}
\usepackage[vcentermath]{youngtab}
\newlabel{fig2:a}{{I(a)}{1}}
\newlabel{fig2:b}{{I(b)}{1}}
\newlabel{fig:i}{{II(c)}{1}}
\newlabel{fig:ii}{{II(d)}{1}}
\newlabel{fig:iii}{{II(e)}{1}}
\newlabel{fig3:c}{{III(f)}{1}}
\newlabel{fig3:d}{{III(g)}{1}}
\newlabel{fig4:i}{{IV(h)}{1}}
\newlabel{fig4:ii}{{IV(i)}{1}}
\newcommand{\Q}{\mathbb{Q}}
\newcommand{\N}{\mathbb{N}}
\newcommand{\R}{\mathbb{R}}
\newcommand{\row}{\mathfrak{row}}
\newcommand{\oh}{\operatorname{outerhook}}
\newcommand{\col}{\mathfrak{col}}
\newcommand{\lis}{\mathfrak{lis}}
\newcommand{\lds}{\mathfrak{lds}}
\newcommand{\sss}{\scriptstyle} %% for tableaux
\newcommand{\swap}{\operatorname{swap}}
\newcommand{\jdt}{\operatorname{jdt}}
\newcommand{\std}{\operatorname{std}}
%\usepackage{fancybullets}
%\usepackage{other packages you may want to use}


\usepackage{amsxtra}
\usepackage{amscd}
\usepackage{amsthm}
\usepackage{amsfonts}
\usepackage{amssymb}
\usepackage{eucal}
\usepackage[all]{xy}
\usepackage{graphicx}
\usepackage{tikz-cd}
\usepackage{mathrsfs}
\usepackage{euler}
\usepackage{hyperref}
\usepackage{color}
\usepackage{longtable}
\usepackage{float}
\usepackage{caption}

\usepackage[colorinlistoftodos, textsize=tiny]{todonotes}
\def\listtodoname{List of Todos}
\def\listoftodos{\@starttoc{tdo}\listtodoname}

\RequirePackage{color}
\definecolor{myred}{rgb}{0.75,0,0}
\definecolor{mygreen}{rgb}{0,0.5,0}
\definecolor{myblue}{rgb}{0,0,0.65}

\usepackage{tikz}
\usepackage{tikz-cd}
\usetikzlibrary{matrix,arrows,decorations.pathmorphing}

%%%%%%%%%%%%%%%%%%%%%%% amsthm theorem styles %%%%%%%%%%%%%%%%%%%%%%%


\theoremstyle{plain}
  \newtheorem{thm}{Theorem}
  \newtheorem{prop}[thm]{Proposition}
  \newtheorem{lem}[thm]{Lemma}
  \newtheorem{cor}[thm]{Corollary}
	\newtheorem{question}[thm]{Question}
	
\theoremstyle{definition}
  \newtheorem{defn}[thm]{Definition}
  \newtheorem{example}[thm]{Example}
	
\theoremstyle{remark}
	\newtheorem{rem}[thm]{Remark}
	

%%%%%%%%%%%%%%%%%%%%%%%%% custom commands %%%%%%%%%%%%%%%%%%%%%%%%%%%

\newcommand\ssec{\subsection}
\newcommand\sssec{\subsubsection}
\newcommand\BH{{\mathbb H}}
\newcommand\BN{{\mathbb N}}
\newcommand\BC{{\mathbb C}}
\newcommand\BR{{\mathbb R}}
\newcommand\BQ{{\mathbb Q}}
\newcommand\BP{{\mathbb P}}
\newcommand\BZ{{\mathbb Z}}
\newcommand\Bk{{\Bbbk}}

\newcommand\sco{{\mathscr O}}

\newcommand\im{\text{im }}
\newcommand\proj{\text{Proj }}

\DeclareMathOperator{\ord}{ord}
\DeclareMathOperator{\sym}{Sym}

%%%%%%%%%%%%%%%%%%%%% cring custom commands %%%%%%%%%%%%%%%%%%%%%%%%%

\DeclareMathOperator\di{Div}
\newcommand\sx{\mathscr X}
\newcommand \subhalf[1]{\frac{{#1} - 1}{2{#1}}}
\newcommand{\halfcan}{L}
\DeclareMathOperator{\Supp}{Supp}
\DeclareMathOperator{\initial}{in_\prec}
\DeclareMathOperator{\Eff}{Eff}
\DeclareMathOperator{\sat}{sat}
\DeclareMathOperator{\newspan}{span}
\captionsetup[table]{belowskip = 4pt}


%\definecolor{BoxCol}{cmyk}{0.9,0.9,0.9}
% uncomment for grey background to \section boxes 
% for use with default option boxedsections

\definecolor{mainCol}{rgb}{1,1,1} %background
\definecolor{BoxCol}{rgb}{.1,.1,.3} 
\definecolor{TextCol}{rgb}{0.2,0.5,0.5}
\definecolor{SectionCol}{rgb}{1,1,1} %section header
\definecolor{ShadeCol}{rgb}{.9,.95,.9}
\definecolor{lightpurple}{rgb}{.1,.1,.3} %title box shade
\definecolor{navy}{rgb}{0.2,0.3,0.5} 	
\definecolor{white}{rgb}{1,1,1}
\def\NN{{\mathbb{N}}}
\def\RR{{\mathbb{R}}}
\newcommand{\Z}{\mathbb{Z}} 
%\definecolor{BoxCol}{rgb}{0.9,0.9,1}
% uncomment for light blue background to \section boxes 
% for use with default option boxedsections
 
%\definecolor{SectionCol}{rgb}{0,0,0.5}
% uncomment for dark blue \section text 


\title{\vspace{.5cm} \textcolor{white} {Spin Canonical Rings of Log Stacky Curves}}

% Note: only give author names, not institute
\author{\textcolor{white}{Aaron Landesman, Peter Ruhm, and Robin Zhang}}

%\email[Aaron Landesman]{aaronlandesman@college.harvard.edu}

% insert correct institute name
\institute{\vspace{-.25cm} \textcolor{white}{Harvard University, Stanford University, Stanford University}}

%\date is unused by the current \maketitle

%\leftlogo[0.9]{nsf_logo.png}  % defines logo to left of title (with scale factor)
%\rightlogo[1.0]{umnlogo.png} % same but on right

%%%%%%%%%%%%%%%%%%%%%%%%%%%%%%%%%%%%%%%%%%%%%%%%%%%%%%%%%%%%%%%%%%%%%%%%%%


\begin{document}
%*** facultative: where the poster was presented (appears as a left footer):
  %\conference{LSU Discover Research Day, 2014, Baton Rouge, LA}


%*** print the poster header defined above: title, authors, affiliations:
 \fbox{\colorbox{lightpurple}{\maketitle}}
\\
\\
  \begin{multicols}{3}
	  \begin{tcolorbox}
	  \section*{Abstract}

 	Consider modular forms arising from a finite-area quotient of the
	upper-half plane by a Fuchsian group. By the classical results of
	Kodaira--Spencer, this ring of modular forms may be viewed as the
	log spin canonical ring of a stacky curve. We
	tightly bound the degrees of minimal generators and relations of
	log spin canonical rings. As a consequence, we obtain a tight bound
	on the degrees of minimal generators and relations for rings of
	modular forms of arbitrary integral weight.
\end{tcolorbox}

  \section*{\textsc{Notation}}
  \begin{align*}
	  \Bbbk & & \text{Perfect Field} \\
	  \sx & & \text{Stacky Curve} \\
	  X & & \text{Coarse Space} \\
	  g & & \text{Genus} \\
	  \Delta & & \text{Log Divisor} \\
	  \delta & & \text{Degree of Log Divisor} \\
	  P_1, \ldots, P_r \in \text{Spec } X & & \text{Stacky Points} \\
	  e_1, \ldots, e_r \in \BZ_{\geq 2}  & & \text{Stacky Orders} \\
	  (g; e_1,\ldots, e_r; \delta) & & \text{Signature} \\
	\left\lfloor \sum_{i=1}^n \alpha_i P_i \right\rfloor := \sum_{i=1} \lfloor \alpha_i\rfloor P_i & & \text{Floor of a Stacky Divisor} \\
	K_X & & \text{Canonical Divisor of Coarse Space} \\
	K_\sx := K_X + \sum_{i=1}^n \frac{e_i-1}{e_i} P_i & & \text{Canonical Divisor of Stacky Curve} \\
	(\sx, \Delta, L), 2L \sim K_\sx + \Delta & & \text{Log Spin Curve} \\
	R(\sx, \Delta, \halfcan) := \bigoplus_{k \geq 0} H^0(X, \lfloor k \halfcan \rfloor) & & \text{Log Spin Canonical Ring}
  \end{align*}

  \section*{\textsc{Introduction}}

  
Let $X$ be a smooth proper geometrically-connected algebraic curve of genus $g$ over a field $\Bk$.
It is well known that the canonical sheaf $\Omega _X,$ with
associated canonical divisor $K_X$, determines the {\bf canonical
map } $\pi: X \rightarrow \BP_\Bk^{g - 1}$. Then, the {\bf canonical
ring} is defined to be
\begin{align*}
	R(X, K_X) := \bigoplus_{d \geq 0} H^0(X, dK_X),
\end{align*}

\noindent
with multiplication structure corresponding to tensor product of
sections. In the case that $g \geq 2$, $\Omega_X$ is ample and
therefore $X \cong \proj R$. When $g \geq 2$, Petri's theorem 
shows that, in most cases, $R(X, K_X)$
is generated in degree 1 with relations in degree 2 (see
Saint-Donat \cite[p. 157]{saint-donat:proj} and Arbarello--Cornalba--Griffiths--Harris
\cite[Section 3.3]{acgh:algebraic-curves}). This has the pleasant 
geometric consequence that canonically embedded curves of genus $\geq 4$ which are not hyperelliptic curves, trigonal curves, or plane quintics are scheme-theoretically cut out by degree 2 equations.

Our main theorem is to bound the degrees of generators and
relations of a log spin canonical ring. In 
~\cite[Theorem 1.4]{vzb:stacky}, Voight and Zureick-Brown tightly
bounde the degree of generation and
relations of a log spin canonical ring.



  \section*{\textsc{Main Result}}


\begin{thm}
\label{thm:main}
Let $(\sx, \Delta, \halfcan)$ be a log spin curve over a perfect
field $\Bk$, so that $\sx$ has signature $\sigma = (g; e_1, \ldots,
e_r; \delta)$.

Then the log spin canonical ring is generated as a $\Bk$-algebra by 
elements of degree at most $e := \max(5, e_1, \ldots, e_r)$ with
relations generated in degrees at most $2e$,
so long as $\sigma$ does not lie in a finite list of exceptional
cases, as given in Table ~\ref{table:g-1-exceptional} for
signatures with $g = 1$ and Table ~\ref{table:g-0-exceptional} for
signatures with $g = 0$.
\end{thm}
{\bf Proof idea:} First, check this holds when $\Delta \in \di X$,
either by hand in $g = 0, 1$ or by 
\cite[$\mathsection$ 6]{milnor:fractional-weight} when $g \geq 2$.
Then, inductively add in fractional points, showing that if
it holds for a given divisor $\Delta$ it holds for
$\Delta + \frac{e-1}{2e}P$.


\begin{table}
\captionsetup{type=figure}
\begin{center}
\begin{tabular}
	{| c || c | c | c |}
	\hline
	Signature $\sigma$ & Generator Degrees & Degrees of Relations & $e$ \\
	\hline
	\hline

	$(0; 3, 3, 3; 0)$ & $\{3\}$ & $\emptyset$ & $5$ \\	\hline

	$(0; 3, 3, 5; 0)$ & $\{3, 10, 15\}$ & $\{30\}$ & $5$ \\	\hline
	
	$(0; 3, 3, 7; 0)$ & $\{3, 7, 12\}$ & $\{24\}$ & $7$ \\	\hline
	
	$(0; 3, 3, 9; 0)$ & $\{3, 7, 9\}$ & $\{21\}$ & $9$ \\	\hline
	
	$(0; 3, 5, 5; 0)$ & $\{3, 5, 10\}$ & $\{20\}$ & $5$ \\	\hline
	
	$(0; 3, 5, 7; 0)$ & $\{3, 5, 7\}$ & $\{17\}$ & $7$ \\	\hline
	
	$(0; 5, 5, 5; 0)$ & $\{3, 5, 5\}$ & $\{15\}$ & $5$ \\	\hline
	
	$(0; 3, 3, 3, 3; 0)$ & $\{3, 3, 4\}$ & $\{12\}$ & $5$ \\	\hline
\end{tabular}
\end{center}
\caption{Genus 0 Exceptional Cases}
\label{table:g-0-exceptional}
\end{table}



\begin{table}	
\captionsetup{type=figure}
\begin{center}
\begin{tabular}
{| c || c | c | c |}
	\hline
	$\halfcan'$ & Generator Degrees & Degrees of Minimal Relations & $e$ \\
	\hline
	\hline
	$P - Q$ & $\{2\}$ & $\emptyset$ & $1$\\	\hline

	$P - Q + \frac{1}{3} P_1$ & $\{2, 3, 7\}$ & $\{14\}$ & $5$ \\	\hline

	$P - Q + \frac{2}{5} P_1$ & $\{2, 3, 5\}$ & $\{12\}$ & $5$\\	\hline
	
	$\frac{1}{3} P_1$ & $\{1, 6, 9\}$ & $\{18\}$ & $5$ \\	\hline

	$\frac{2}{5} P_1$ & $\{1, 5, 8\}$ & $\{16\}$ & $5$ \\	\hline
	
	$\frac{3}{7} P_1$ & $\{1, 5, 7\}$ & $\{15\}$ & $7$ \\	\hline
\end{tabular}

\end{center}
\caption{Genus 1 Exceptional Cases}
\label{table:g-1-exceptional}
\end{table}


\section*{Applications to Modular Forms}

\begin{cor}
\label{cor:main-mod-forms}
Let $\Gamma$ be a Fuchsian, i.e., $\Gamma \backslash \BH$ has finite area, group and $\sx$ the stacky curve
associated to $\Gamma \backslash \BH$ with signature $\sigma
= (g; e_1, \ldots, e_r; \delta)$. 

Except in an explicit finite list of cases, the
ring of modular forms $M(\Gamma)$ is generated as a $k$-algebra
by elements of degree at most $\max(5, e_1, \ldots, e_r)$ with
relations generated in degree at most $2 \cdot \max(5, e_1, \ldots, 
e_r)$.
\end{cor}

\begin{example}
\label{eg:congruence-bounds}
Suppose $\Gamma \subset SL_2(\BZ)$ is any congruence subgroup.
Then $\Gamma$ is generated in weight at most $6$ with relations in weight at most $12$. Further, if  $\Gamma$ has some nonzero odd weight modular form, $M(\Gamma)$ is 
generated in degree at most $5$ with relations in degree at most $10$. 
If additionally, the genus of the stacky curve associated to $\Gamma \backslash \BH^*$ is 0 or 1, then $M(\Gamma)$ is generated in weight at most $4$ with
relations in weight at most $8$.\end{example}


\section*{Acknowledgments}
We are grateful to David Zureick-Brown for introducing us to the
study of stacky canonical rings, for providing invaluable guidance,
and for his mentorship. We also thank Ken Ono for arranging our project. We thank Brian Conrad, John Voight, and Shou-Wu
Zhang, and Jorge Neves
for helpful comments. Finally, we
acknowledge the support of the National Science Foundation (grant
number DMS-1250467).
%%%%%%%%%%%%%%%%%%%%%%%%%%%% References %%%%%%%%%%%%%%%%%%%%%%%%%%%%%%%
\todo{Remove some of these references}
\nocite{*}
\bibliography{bibliography}{}
\bibliographystyle{plain} 


\end{multicols}
\end{document}

