%        File: changes-summary.tex
%     Created: Sat Oct 17 11:00 AM 2015 E
% Last Change: Sat Oct 17 11:00 AM 2015 E
%
%%%%%%%%%%%%%%%%%%%%%
%   AMS packages    %
%%%%%%%%%%%%%%%%%%%%%
\documentclass[10 pt]{amsart}

\usepackage{amsmath}
\usepackage{amsxtra}
\usepackage{amscd}
\usepackage{amsthm}
\usepackage{amsfonts}
\usepackage{amssymb}
\usepackage{eucal}
\usepackage[all]{xy}
\usepackage{graphicx}
\usepackage{tikz-cd}
\usepackage{mathrsfs}
\usepackage{subfiles}
\usepackage{mathpazo}
\usepackage{euler}
\usepackage[colorinlistoftodos, textsize=tiny]{todonotes}

\addtolength{\oddsidemargin}{-.5 in}
\addtolength{\evensidemargin}{-.4 in}
\addtolength{\textwidth}{1 in}
\addtolength{\topmargin}{0 in}
\addtolength{\textheight}{0 in}

\RequirePackage{color}
\definecolor{myred}{rgb}{0.75,0,0}
\definecolor{mygreen}{rgb}{0,0.5,0}
\definecolor{myblue}{rgb}{0,0,0.65}

\usepackage{hyperref}
\hypersetup{colorlinks=true,citecolor=blue}
\usepackage{tikz}
\usetikzlibrary{matrix,arrows,decorations.pathmorphing}

\theoremstyle{plain}
\newtheorem{theorem}{Theorem}[section]
\newtheorem{proposition}[theorem]{Proposition}
\newtheorem{lemma}[theorem]{Lemma}
\newtheorem{corollary}[theorem]{Corollary}
\theoremstyle{definition}
\newtheorem{definition}[theorem]{Definition}
\newtheorem{remark}[theorem]{Remark}
\newtheorem{example}[theorem]{Example}
\newtheorem{exercise}[theorem]{Exercise}
\newtheorem{counterexample}[theorem]{Counterexample}
\newtheorem{convention}[theorem]{Convention}
\newtheorem{question}[theorem]{Question}
\newtheorem{conjecture}[theorem]{Conjecture} 
\newtheorem{warning}[theorem]{Warning}
\newtheorem{fact}[theorem]{Fact}
\theoremstyle{remark}
\newtheorem{notation}[theorem]{Notation}
\numberwithin{equation}{section}
  
\newcommand\nc{\newcommand}
\nc\on{\operatorname}
\nc\renc{\renewcommand}
\newcommand\se{\section}
\newcommand\ssec{\subsection}
\newcommand\sssec{\subsubsection}
\newcommand\bn{{\mathbb N}}
\newcommand\bc{{\mathbb C}}
\newcommand\br{{\mathbb R}}
\newcommand\bq{{\mathbb Q}}
\newcommand\bp{{\mathbb P}}
\newcommand\CF{{\mathcal F}}
\newcommand\bz{{\mathbb Z}}
\newcommand\ba{{\mathbb A}}
\newcommand\fa{{\mathfrak a}}
\newcommand\fp{{\mathfrak p}}
\newcommand\fq{{\mathfrak q}}
\newcommand\fm{{\mathfrak m}}
\newcommand\so{{\mathscr O}}
\newcommand\sg{{\mathscr G}}

\newcommand\scm{{\mathscr M}}
\newcommand\scn{{\mathscr N}}
\newcommand\scf{{\mathscr F}}
\newcommand\scg{{\mathscr G}}
\newcommand\sco{{\mathscr O}}
\newcommand\sch{{\mathscr H}}
\newcommand\scl{{\mathscr L}}
\newcommand\sci{{\mathscr I}}

\newcommand \ra{\rightarrow}
\newcommand{\id}{\mathrm{id}}
\newcommand\im{\text{im }}
\newcommand\coker{\text{coker}}
\newcommand \spec{\text{Spec }}
\newcommand \proj{\text{Proj }}
\newcommand \rspec{\textit{Spec }}
\newcommand \rproj{\textit{Proj }}
\newcommand \mg{{\mathscr M_g}}

\DeclareMathOperator\ord{ord}

\newcommand \trdeg{\text{tr. deg }}
\newcommand \codim{\text{codim}}
\newcommand \rk{\text{rk }}
\newcommand \di{\text{div }}
\newcommand \depth{\text{depth }}
\DeclareMathOperator\pic{Pic}
\DeclareMathOperator\lcm{lcm}
\DeclareMathOperator\rank{rank}
\DeclareMathOperator\vol{Vol}
\DeclareMathOperator\supp{Supp}

\usepackage{xr}
\externaldocument{../surface-paper}


\def\listtodoname{List of Todos}
\def\listoftodos{\@starttoc{tdo}\listtodoname}

\title{Summary of Changes in response to Referee's Report}
\author{Aaron Landesman, Peter Ruhm, and Robin Zhang}

\begin{document}

\maketitle
\section{Changes Made}
Here are some responses to the referee's suggestions.
Item's below are numbered to match the referee report.
We applied nearly all changes suggested by the referee, though most need no
comment. Here, we make some brief comments on how we applied certain suggested
changes.
\begin{enumerate}
	\item[(2)] 
We have now added in a detailed example 
\autoref{example:effective-example}
explaining how to apply our method of proof to actually work out
the generators and relations for a section ring
associated to an effective divisor on $\bp^m$.
\item[(10)]
Response to the concern about base change along inseparable extensions:
We added Remark \autoref{rem:algebraically-closed}
explaining why purely inseparable extensions pose
no addition issue. The point is that cohomology commutes with flat base change, and extensions of fields
are always flat, so the dimensions of the graded pieces of the ring will be preserved. Therefore generators
and relations are preserved under arbitrary base field extension. Of course,
canonical divisors of root stacks may not behave well with respect to base change (as we learned from a version of Voight--Zureick-Brown's forthcoming book available on Voight's website)
which is why inseparable extensions
may cause problems for canonical rings.
But, for the purposes of this article, we fix a divisor on a scheme over some base field, and the structure of that particular section
ring is unchanged upon base change to the algebraic closure.

\item[(14) and (18)]
We explained in both the projective space case and Hirzebruch case that this assumption was nonessential, adding
\autoref{cor:proj-generators-relations} for the projective case
and \autoref{rem:hirz-generators-relations-hypothesis} for the Hirzebruch case.
You are correct that we could have easily changed coordinates and reordered the $f_i$ so that the first $m+1$ $f_i$ were $x_0, \ldots, x_m$, but
it seems we may as well state it with arbitrary independent polynomials, so as not to require such a change of basis.
\item[(22)]
	We moved the notation to its own section after the introduction,
	as most of the notation was not necessary to state
	our main theorems.
\item[(24)]
	You're right that the proof of \autoref{prop:pm-span-and-basis}
		was poorly phrased, and we have tried to clarify this now.
The point is that we are only varying $f_0, \ldots, f_m,$ so
the resulting monomials in the $f_i$ will be independent because products of the $f_i$ are. See the updated proof in the paper for the details.
\item[(25)]
It is not an ordering, only a partial ordering (which we have now stated).
\item[(27)]
Good point, we should have done a reduction instead of claiming it was analogous.
\item[(28)]
Wouldn't there be something like $\binom{m + c_1- 1}{c_1}$ elements? For example, if
we take $\alpha = 1$, and $H = V(x_0)$ we will have generators $x_1/x_0, \ldots, x_m/x_0$.
In any case, we decided to remove the paragraph this comment
was referring to and the claim about minimality
of the presentation from the theorem.
\end{enumerate}
\end{document}


