%        File: changes-summary.tex
%     Created: Sat Oct 17 11:00 AM 2015 E
% Last Change: Sat Oct 17 11:00 AM 2015 E
%
%%%%%%%%%%%%%%%%%%%%%
%   AMS packages    %
%%%%%%%%%%%%%%%%%%%%%
\documentclass[10 pt]{amsart}

\usepackage{amsmath}
\usepackage{amsxtra}
\usepackage{amscd}
\usepackage{amsthm}
\usepackage{amsfonts}
\usepackage{amssymb}
\usepackage{eucal}
\usepackage[all]{xy}
\usepackage{graphicx}
\usepackage{tikz-cd}
\usepackage{mathrsfs}
\usepackage{subfiles}
\usepackage{mathpazo}
\usepackage{euler}
\usepackage[colorinlistoftodos, textsize=tiny]{todonotes}

\addtolength{\oddsidemargin}{-.5 in}
\addtolength{\evensidemargin}{-.4 in}
\addtolength{\textwidth}{1 in}
\addtolength{\topmargin}{0 in}
\addtolength{\textheight}{0 in}

\RequirePackage{color}
\definecolor{myred}{rgb}{0.75,0,0}
\definecolor{mygreen}{rgb}{0,0.5,0}
\definecolor{myblue}{rgb}{0,0,0.65}

\usepackage{hyperref}
\hypersetup{colorlinks=true,citecolor=blue}
\usepackage{tikz}
\usetikzlibrary{matrix,arrows,decorations.pathmorphing}

\theoremstyle{plain}
\newtheorem{theorem}{Theorem}[section]
\newtheorem{proposition}[theorem]{Proposition}
\newtheorem{lemma}[theorem]{Lemma}
\newtheorem{corollary}[theorem]{Corollary}
\theoremstyle{definition}
\newtheorem{definition}[theorem]{Definition}
\newtheorem{remark}[theorem]{Remark}
\newtheorem{example}[theorem]{Example}
\newtheorem{exercise}[theorem]{Exercise}
\newtheorem{counterexample}[theorem]{Counterexample}
\newtheorem{convention}[theorem]{Convention}
\newtheorem{question}[theorem]{Question}
\newtheorem{conjecture}[theorem]{Conjecture} 
\newtheorem{warning}[theorem]{Warning}
\newtheorem{fact}[theorem]{Fact}
\theoremstyle{remark}
\newtheorem{notation}[theorem]{Notation}
\numberwithin{equation}{section}
  
\newcommand\nc{\newcommand}
\nc\on{\operatorname}
\nc\renc{\renewcommand}
\newcommand\se{\section}
\newcommand\ssec{\subsection}
\newcommand\sssec{\subsubsection}
\newcommand\bn{{\mathbb N}}
\newcommand\bc{{\mathbb C}}
\newcommand\br{{\mathbb R}}
\newcommand\bq{{\mathbb Q}}
\newcommand\bp{{\mathbb P}}
\newcommand\CF{{\mathcal F}}
\newcommand\bz{{\mathbb Z}}
\newcommand\ba{{\mathbb A}}
\newcommand\fa{{\mathfrak a}}
\newcommand\fp{{\mathfrak p}}
\newcommand\fq{{\mathfrak q}}
\newcommand\fm{{\mathfrak m}}
\newcommand\so{{\mathscr O}}
\newcommand\sg{{\mathscr G}}

\newcommand\scm{{\mathscr M}}
\newcommand\scn{{\mathscr N}}
\newcommand\scf{{\mathscr F}}
\newcommand\scg{{\mathscr G}}
\newcommand\sco{{\mathscr O}}
\newcommand\sch{{\mathscr H}}
\newcommand\scl{{\mathscr L}}
\newcommand\sci{{\mathscr I}}

\newcommand \ra{\rightarrow}
\newcommand{\id}{\mathrm{id}}
\newcommand\im{\text{im }}
\newcommand\coker{\text{coker}}
\newcommand \spec{\text{Spec }}
\newcommand \proj{\text{Proj }}
\newcommand \rspec{\textit{Spec }}
\newcommand \rproj{\textit{Proj }}
\newcommand \mg{{\mathscr M_g}}

\DeclareMathOperator\ord{ord}

\newcommand \trdeg{\text{tr. deg }}
\newcommand \codim{\text{codim}}
\newcommand \rk{\text{rk }}
\newcommand \di{\text{div }}
\newcommand \depth{\text{depth }}
\DeclareMathOperator\pic{Pic}
\DeclareMathOperator\lcm{lcm}
\DeclareMathOperator\rank{rank}
\DeclareMathOperator\vol{Vol}
\DeclareMathOperator\supp{Supp}

\usepackage{xr}
\externaldocument{../surface-write-up}


\def\listtodoname{List of Todos}
\def\listoftodos{\@starttoc{tdo}\listtodoname}

\title{Summary of Changes in response to Referee's Report}
\author{Aaron Landesman, Peter Ruhm, and Robin Zhang}

\begin{document}

\maketitle
\section{Changes Made}
Here are the changes we have made in response to the referee's suggestions.
\todo{make sure the theorem number here reflect the new numberings}
\begin{enumerate}
	\item \todo{Ask David how the new version sounds} We explained
		in more detail how our potential application
		for Hilbert modular forms and the difficulty in producing
		an explicit example.
	\item In all of the main theorems, it says that $\frac{c_i}{k_i}$ is in reduced form. 
	\item In Theorem \autoref{thm:proj-generators-relations}, $k$ is a field. Throughout our paper, the results hold for arbitrary fields, but we may assume the field is algebraically closed because the generator and relation degrees are preserved by base change to the algebraic closure. We have now included and clarified this in the ``Notations and Conventions'' subsection at the end of the Introduction.
	\item On page 2, line -9 of the previous version, $\deg(D)$ should have been $\frac{1}{2 \prod_{i=0}^n q_i}$ and is now corrected.
	\item Theorem 1.4 (now \autoref{thm:hirz-generators-relations}) has been rewritten into a simpler form by reducing some
	notation unnecessary to understanding the general idea.
	\item Subsection 1.2 has now been moved to immediately precede Subsection 1.1 as a standalone paragraph.
	\item The reference to later items given on page 5, line 7 of the previous version
		has been moved to Remark \autoref{rem:proj-grobner}.
\item The use of best lower approximations has been replaced by the use of
convergents of Hirzebruch-Jung continued fractions which are equivalent and
well-known.
\item The references for the O'Dorney and Landesman--Ruhm--Zhang papers have been updated. The citation for arxiv articles now includes the arxiv reference numbers.
\item The conditions of Proposition~\autoref{prop:pm-span-and-basis} have been reorganized for clarity.
\item Remark 1.3 has been split into what is now Remark~\autoref{rem:proj-tight-bounds-eff} and Remark~\autoref{rem:proj-tight-bounds} (with
some material moved to Remark~\autoref{rem:exact-noneff-bounds}).

\end{enumerate}



\end{document}


