%%%%%%%%%%%%%%%%%%%%%
%   AMS packages    %
%%%%%%%%%%%%%%%%%%%%%
\documentclass{amsart}

\usepackage{amsmath}
\usepackage{amsxtra}
\usepackage{amscd}
\usepackage{amsthm}
\usepackage{amsfonts}
\usepackage{amssymb}
\usepackage{eucal}
\usepackage[all]{xy}
\usepackage{graphicx}
\usepackage{tikz-cd}
\usepackage{mathrsfs}
\usepackage{subfiles}
%\usepackage{mathpazo} not a huge fan
\usepackage{euler}
\usepackage{hyperref}
\usepackage{color}
\usepackage{longtable}
\usepackage{float}
\usepackage{caption}

\usepackage[colorinlistoftodos, textsize=tiny]{todonotes}
\def\listtodoname{List of Todos}
\def\listoftodos{\@starttoc{tdo}\listtodoname}

%\addtolength{\oddsidemargin}{-.5 in}
	%\addtolength{\evensidemargin}{-.4 in}
	%\addtolength{\textwidth}{1 in}
%
	%\addtolength{\topmargin}{0 in}
	%\addtolength{\textheight}{0 in}

\RequirePackage{color}
\definecolor{myred}{rgb}{0.75,0,0}
\definecolor{mygreen}{rgb}{0,0.5,0}
\definecolor{myblue}{rgb}{0,0,0.65}

%\usepackage{hyperref}
  %\hypersetup{colorlinks=true,citecolor=blue}

\usepackage{tikz}
\usepackage{tikz-cd}
\usetikzlibrary{matrix,arrows,decorations.pathmorphing}

%%%%%%%%%%%%%%%%%%%%%%% amsthm theorem styles %%%%%%%%%%%%%%%%%%%%%%%

\theoremstyle{plain}
  \newtheorem{thm}{Theorem}[section]
  \newtheorem{prop}[thm]{Proposition}
  \newtheorem{lem}[thm]{Lemma}
  \newtheorem{cor}[thm]{Corollary}
	\newtheorem{claim}[thm]{Claim}
	\newtheorem{question}[thm]{Question}
	
\theoremstyle{definition}
  \newtheorem{defn}[thm]{Definition}
  \newtheorem{example}[thm]{Example}
  \newtheorem{exer}[thm]{Exercise}
  \newtheorem{ctexample}[thm]{Counterexample}
  \newtheorem{convention}[thm]{Convention}
	\newtheorem{conjecture}[thm]{Conjecture}
	
\theoremstyle{remark}
	\newtheorem{rem}[thm]{Remark}
  \newtheorem{note}[thm]{Notation}
  \newtheorem*{note*}{Notation}
  \newtheorem{case}{Case}
	
\numberwithin{equation}{section}

%%%%%%%%%%%%%%%%%%%%%%%%% custom commands %%%%%%%%%%%%%%%%%%%%%%%%%%%

\newcommand\nc{\newcommand}
\nc\on{\operatorname}
\nc\renc{\renewcommand}
\newcommand\ssec{\subsection}
\newcommand\sssec{\subsubsection}
\newcommand\bh{{\mathbb H}}
\newcommand\bn{{\mathbb N}}
\newcommand\bc{{\mathbb C}}
\newcommand\bbf{{\mathbb F}}
\newcommand\br{{\mathbb R}}
\newcommand\bq{{\mathbb Q}}
\newcommand\bp{{\mathbb P}}
\newcommand\bz{{\mathbb Z}}
\newcommand\ba{{\mathbb A}}
\newcommand\bk{{\Bbbk}}

\newcommand\sco{{\mathscr O}}

\newcommand{\id}{\mathrm{id}}
\newcommand\im{\text{im }}
\newcommand\coker{\text{coker}}
\newcommand\gal{\mathrm{Gal}}

\DeclareMathOperator{\ord}{ord}
\DeclareMathOperator{\sym}{Sym}

%%%%%%%%%%%%%%%%%%%%% csurfaces custom commands %%%%%%%%%%%%%%%%%%%%%%%%%

\DeclareMathOperator\di{Div}
\newcommand\bida{a}
\newcommand\bidb{b}
\newcommand\pdeg{\delta}
\newcommand\hirz{\mathbb{F}}
\newcommand\mss{\mathscr{S}}

\DeclareMathOperator{\fr}{frac}
\DeclareMathOperator{\supp}{Supp}
\DeclareMathOperator{\initial}{in_\prec}
\DeclareMathOperator{\gin}{gin}
\DeclareMathOperator{\Eff}{Eff}
\DeclareMathOperator{\sat}{sat}
\DeclareMathOperator{\newspan}{span}
\DeclareMathOperator{\proj}{Proj}
\DeclareMathOperator{\spec}{Spec}
\DeclareMathOperator{\Te}{T_=}
\DeclareMathOperator{\Tp}{T_+}
\DeclareMathOperator{\Tm}{T_-}
\DeclareMathOperator{\num}{p}
\DeclareMathOperator{\den}{q}
\DeclareMathOperator{\lcm}{lcm}
\DeclareMathOperator{\Pic}{Pic}
%\captionsetup[table]{belowskip = 4pt}

\makeatletter
\newcommand{\customlabel}[2]{%
   \protected@write \@auxout {}{\string \newlabel {#1}{{#2}{\thepage}{#2}{#1}{}} }%
   \hypertarget{#1}{#2}
}
\makeatother

%%%%%%%%%%%%%%%%%%%%%%%%%%%%%% title %%%%%%%%%%%%%%%%%%%%%%%%%%%%%%%%

\title{Canonical rings of $\bq$-divisors on Minimal Rational Surfaces}

\author{Aaron Landesman}
\address[Aaron Landesman]{Department of Mathematics, Harvard University}
\email{aaronlandesman@college.harvard.edu}

\author{Peter Ruhm}
\address[Peter Ruhm]{Department of Mathematics, Stanford University}
\email{pruhm@stanford.edu}

\author{Robin Zhang}
\address[Robin Zhang]{Department of Mathematics, Stanford University}
\email{robinz16@stanford.edu}

\date{\today}

%%%%%%%%%%%%%%%%%%%%%%%%%%%%% document %%%%%%%%%%%%%%%%%%%%%%%%%%%%%%

\begin{document}

\begin{abstract}
 	Bounding generator and relation degrees of general $\bq$-divisors
	on projective spaces $\bp^n$ and Hirzebruch surfaces.
\end{abstract}

\maketitle
\todo{make title shorter so it doesn't overlap with page numbers}
\todo{Aaron: Standardize generation in degree ... phrasing. I like ``generation in degree at most''}

%%%%%%%%%%%%%%%%%%%%%%%%%%%% Introduction %%%%%%%%%%%%%%%%%%%%%%%%%%%%%%%

\section{Introduction}
\label{sec:intro}
For any Weil $\bq$-divisor $D$ on a rational surface $X$, the graded
\textbf{canonical ring} $R(X, D) := \bigoplus_{d \geq 0} H^0(X, dD)$
has connections to rings of Hilbert and Siegel modular forms. In the
study of these canonical rings, it is natural to describe
presentations. We study the generators and relations of canonical
rings for projective spaces $\bp^m$ and Hirzebruch surfaces $\hirz_m$.
When $D$ is a general $\bq$-divisor on any $\bp^m$ or on any
$\hirz_m$, we give bounds on the generators and relations of
$R(X, D)$. In particular, we give a complete description of the
canonical ring when $X = \bp^m$ and $D$ is an $\bq$-divisor
containing one hypersurface and give stronger bounds when
$D$ is any effective $\bq$-divisor on $\bp^m$.

In part, this work is motivated by moving to generalize the work
of O'Dorney \cite{dorney:canonical} who gives similar descriptions
of canonical rings for general $\bq$-divisors on $\bp^1$. For
particular $\bq$ divisors on stacky curves, the work by
Voight--Zureick-Brown \cite{vzb:stacky} and Landesman--Ruhm--Zhang
\cite{lrz:spin-cring} on log spin canonical rings provides
useful bounds on rings of modular forms of Fuchsian groups.
The results of this work may have applications to Hilbert
and Siegel modular forms. \todo{reword last sentence}



\todo{Hassett-keel, see reference in davids paper about parametrizing $\alpha$ and considering stacky divisors. Would want to know whether they are a moduli space, though we just give generators and relations. This is one of the few examples other than that of nagata dealing with stacky surfaces.}

The work in this paper also has relations to certain rings of Hilbert 
modular forms and Siegel modular forms. Recall that Hilbert and Siegel 
modular forms are both to higher dimensions variables. In the case of 
Hilbert modular forms, one considers modular forms on a product of $m$
upper half planes, which transform under the action of finite index 
subgroups of the groups of positive determinant matrices. Similarly, Siegel 
modular forms are modular forms on the Siegel upper half space which are 
stable under the action of the symplectic group. 

By a result by Voight and Zureick-Brown \cite[Proposition 6.1.5]{vzb:stacky}, one can translate the analytic category of 1-dimensional complex orbifolds to the algebraic category of stacky curves. 
We expect that  will similarly allow one to generalize yield an equivalence of categories between an appropriate subcategory of $m$-dimensional orbifolds and a subcategory of $m$-dimensional algebraic spaces.
In particular, this will allow us to translate rings of Hilbert modular forms and Siegel modular forms whose underlying space is isomorphic to $\bp^m$ or a Hirzebruch surface to canonical rings on the corresponding algebraic surfaces. Thereby, our techniques would give a bound on generators and relations for Hilbert and Siegel modular forms with certain underlying spaces.
\todo{include references}

Another closely related topic is the Hassett-Keel program 
initiated in \todo{was it actually initiated here? check vzb.} \cite{hassett:classical-and-minimal-models}, which aims to describe log canonical models of the form
\begin{align*}
	\overline {\mathscr M}_g(\alpha) := \bigoplus_{d \geq 0}H^0 \left( \overline {\mathscr M}_g, \lfloor d K_{\overline{\mathscr M}_g} + \alpha\delta \rfloor  \right) 
\end{align*}
in terms of certain moduli spaces. $\bq$-divisors on surfaces also naturally appear when considering the canonical ring of surfaces of Kodaira dimension 1. A surface has Kodaira dimension 1 if and only if it is an elliptic surface, and their canonical ring are most naturally described in the setting of $\bq$-divisors. \todo{Add references}

Furthermore, this work extends the study of canonical rings of
$\bq$-divisors on curves, as done by O'Dorney \cite{dorney:canonical},
Voight--Zureick-Brown \cite{vzb:stacky}, and Landesman--Ruhm--Zhang
\cite{lrz:spin-cring}, to general projective space and Hirzebruch
surfaces.

\todo{more!}

\ssec{Main Results}

\todo{put main results! use restate, maybe?}

\begin{thm}
\label{thm:proj-effective-intro}
Let $X = \bp^m$ with $\{x_0, \ldots, x_m\}$ a coordinate system for
$\bp^m$. Let $D = \sum_{i = 1}^{n} \alpha_i C_i$, with $\alpha_i =
\frac{p_i}{q_i} \in \bq$ and each $C_i$ a hypersurface of
$\bp^m$, be an effective divisor on $\bp^m$.

Then the canonical ring
\[
	R_D := \bigoplus_{d \geq 0} H^0(\bp^m, \lfloor dD \rfloor)
\]

\noindent
is generated in degrees up to $\max_{1 \leq i \leq n}{q_i}$ with
relations generated in degrees up to $2 \max_{1 \leq i \leq n}{q_i}$
\end{thm}

\todo{transition}

\begin{thm}
\label{proj-generators-relations-intro}
Let $D = \sum_{i=1}^n \alpha_i C_i$ for curves $C_i$ on $\mathbb{P}^m$.  Let $f_i$ be homogeneous polynomials in $x_0, \ldots, x_m$ such that $C_i = v(f_i)$ for all $i$; suppose $\deg(f_i) = a_i$. 
Then $R_D$ is generated in degree at most 
\[
	\sum_{i=1}^n \ell_i a_i
\]
with relations generated in degree at most
\[
	\max \left(2 \sum_{i=1}^n \ell_i a_i, \frac{\max_{1\le i \le n}(\bida_i)}{\deg(D)} + \sum_{i=1}^n \ell_i a_i \right).
\]
\end{thm}

\todo{transition}

\begin{thm}
\label{lem:hirz-generators-relations}
Let $D = \sum_{i=1}^n \alpha_i C_i$ for curves $C_i$ on $\hirz_k$.  
Let $C_i = V(f_i)$ where $f_i \in \sco(a_i, b_i)$; further 
suppose $f_1$ and $f_2$ are independent linear
polynomials in $x_0$ and $x_1$ and $f_3$ and $f_4$ are independent
linear polynomials in $y_0$ and $y_1$.
Then $R_D$ is generated in degree less than 
\[
	\sigma = \sum_{i\in \Te(D)} \gcd(\bida_i, \bidb_i)\ell_i +
	\sum_{\substack{i	\in \Tp(D) \\	j \in \Tm(D)}} (\bida_i
	\bidb_j - \bida_j \bidb_i)\ell_{i,j}
\]

\noindent
with relations generated in degree less than 
\[
	\max(2 \sigma, \sigma + \tau)
\]

\noindent
where
\[
	\tau = \sigma
	+ \max \left( \max_{i\in \Te}(\ell_i \gcd(a_i, b_i)), \max_{
	\substack{i \in	\Tp \\ j\in \Tm}}(\ell_{i,j} (\bida_i \bidb_j
	- \bida_j \bidb_i))	\right).
\]
\end{thm}

\todo{Include summary of what each of the sections do.}

\section{Lemmas for Generation and Relations}
\label{sec:lemmas}

\begin{defn}
\label{defn:lower-approximation}
If $\alpha \in \br$, then a rational number $\frac{c}{k} \leq \alpha$
(written in reduced form with $c \in \bz, k \in \bn$) is a
\textbf{best lower approximation} to $\alpha$ if there does not
exist $\frac{c'}{k'}\in \mathbf{Q}$ such that $0 < k'< k$ and
$\frac{c}{k} \le \frac{c'}{k'} \le \alpha$. 
\end{defn}

\begin{rem}
\label{rem:lower-approximation}
Note that all integers less than or equal to $\lfloor \alpha \rfloor$
are best lower approximations to $\alpha$. Also, if $\alpha \ge 0$,
then the non-negative best lower approximations of
$\alpha$ form a finite sequence
\[
	0 = \frac{c_0}{k_0} < \frac{c_1}{k_1} < \ldots < \frac{c_r}{k_r} = \alpha.
\]

\noindent
Figure ~\ref{fig:s14/5-lattice} gives a pictorial representation of the positive best lower approximations of $\alpha = \frac{14}{5}$.
\end{rem}
\todo{possibly define mediant}

\begin{figure}
\includegraphics{pics/spin-lower-approximations-pic-pics.pdf}
\caption{This figure shows each non-negative best lower
approximation of $\frac{14}{5}.$ Each ``$\bullet$'' denotes a best
lower approximation and each ``$\circ$'' denotes a lattice point
below $5y=14x$ which is not a best lower approximation.  Note that
the non-negative best lower approximations generate the monoid of
lattice points in the first quadrant satisfying  $5y \le 14x$, with
the operation $(a_1, b_1)(a_2, b_2)\mapsto (a_1 + a_2, b_1 + b_2)$.}
\label{fig:s14/5-lattice}
\end{figure}

\begin{convention}
Notate a (Weil) $\bq$-divisor
\begin{align*}
	D = \sum_{i=1}^{n}\alpha_i D_i \in \di X \otimes \bq.
\end{align*}

\noindent
where $\alpha_i \in \bq$ and $D_i \in \di X$ is an irreducible divisor.
\end{convention}

Now we give a result describing the generators of a cone via its
extremal rays.
\todo{write more transition}

\begin{prop}
\label{prop:cone-generation}
Let $n \in \bz,$ let $\alpha_1, \ldots, \alpha_n \in \bq$, and let
$a_i, b_i \in \bz$ with $1 \leq i \leq n.$ Define
\begin{align*}
	\Sigma := \left \{(d, c_1, \ldots, c_n) \in \bz^{n + 1} : c_i \geq -
	d \alpha_i, 1 \leq i \leq n \text{ and } \sum_{i = 1}^{n} a_i =
	\sum_{i	= 1}^{n}b_i = 0 \right \}
\end{align*}

\noindent
Suppose $e_1, \ldots, e_t \in \Sigma$ with $e_i = (\pdeg_i, c_1^i,
\ldots, c_n^i)$ are a set of {\bf extremal rays} of $\Sigma$,
i.e.~ $\Sigma$ is contained in the $R_{\geq 0}$ span of
$e_1, \ldots, e_t$.

Then, as a semigroup, $\Sigma$ is generated by
elements whose first coordinate is less than $\sum_{i = 1}^{t}
\pdeg_i$. Furthermore, every element $\sigma \in \Sigma$ can be
written in a canonical form $\lambda + \sum_{i = 1}^{t} w_i e_i$ 
with $\lambda = \sum_{i = 1}^{r} s_i e_i$ with $0 \leq s_i < 1$, the
first coordinate of $\lambda$ less than $\sum_{i=1}^{t}\pdeg_i$,
and $w_1, \ldots, w_t \in \bz_{\geq 0}$.
\end{prop}

\begin{proof}
By assumption, $\sigma \in \Sigma$ can be written as $\sigma = \sum_
{i = 1}^{t} r_i e_i$ with $r_i \in \br$. Let $\fr(r) := r - \lfloor r
\rfloor$ denote the fractional part of $r$. Let $\lambda = \sum_{i = 1}
^{t} \fr(r_i)$. Whence, we can write $\sigma = \lambda + \sum_{i = 1}
^{t} \lfloor r_i \rfloor e_i.$ Consequently, $\sigma$ lies in the
$\bz_{\geq 0}$ span of $\lambda, e_1, \ldots, e_t$, which all have
first coordinate less than $\sum_{i=1}^{t} r_i$. Ergo, $\Sigma$ is
generated by elements whose first coordinate is less than
$\sum_{i = 1}^{t} r_i$.
\end{proof}


\begin{lem}
\label{lem:composite-map}
Let $D = \sum_{i=1}^{n}\alpha_i D_i$ where $D_i = V(f_i)$. Suppose we have a surjection $\phi: \bk[x_1,x_2,\ldots, x_r] \rightarrow R_D$ given by $x_i \mapsto p_i(f_1, \ldots, f_n),$ where $p_i$ is a monomial in $f_1,\ldots, f_n$. Let $\bida_1, \ldots, \bida_n, \bidb_1, \ldots, \bidb_n \in \bz_{\geq 0}.$ Then, define
\begin{align*}
	\Sigma = \langle u^d z_1^{c_1} \cdots z_n^{c_n} : c_i \geq -d \alpha_i, \sum_{i=1}^{n} \bida_i c_i = \sum_{i=1}^{n} \bidb_i c_i = 0 \rangle. 
\end{align*}
In this case, we can factor $\phi$ as a composition of $\chi$ and $\psi$ defined by
\todo{Make this have mapsto on bottom}
\[
\begin{tikzcd}
\bk[x_1,\ldots, x_r] \ar {r}{\chi} & \bk[\Sigma] \ar {r}{\psi} & R_D \\
x_i \ar r & u^{d_i}z_1^{c_{i1}} \cdots z_n^{c_{in}} \ar r & u^{d_i}f_1^{c_{i1}} \cdots f_n^{c_{in}}.
\end{tikzcd}
\]
Assuming $\chi$ is surjective, the minimal degree of generation of $\ker \phi$ is at most the maximum of the minimal degree of generation of $\ker \chi$ and the minimal degree of generation of $\ker \psi$.
\end{lem}
\begin{proof}
First, we show there is an exact sequence
\[
\begin{tikzcd}
0 \ar {r} & \ker \chi \ar{r}{\rho} & \ker \phi \ar {r}{\tau} & \ker \psi \ar r & 0.
\end{tikzcd}
\]
To see this, note that $\rho$ is injective and $\tau$ is surjective\ because $\phi = \chi \circ \psi$. Additionally, $\tau \circ \rho = 0$ by definition of the kernel of $\chi$. Finally, if $a \in\ker \tau$, then by definition $\chi(a) = 0,$ so $a \in \im \rho$.

This shows that lifts of generators of $\ker \psi$ together with images of generators of $\ker \chi$ generate all of $\ker \phi,$ as desired. 
\end{proof}

\begin{lem}
\label{lem:bound-ker-chi}
Retaining the notation of Lemma ~\ref{lem:composite-map}, if $\Sigma$ has extremal rays $e_1,\ldots, e_r$ in degrees $d_1, \ldots, d_r$ then $\ker \chi$ is generated in degrees up to $2(\sum_{i=1}^{r}d_r-1)$.
\end{lem}
\begin{proof}
Since $e_1, \ldots, e_r$ are extremal rays, by Proposition ~\ref{prop:cone-generation}, every element $\sigma \in \Sigma$ can be written in a canonical form $\lambda + \sum_{i=1}^{r}e_i$. Now, let $\lambda_0 := 0,\lambda_1, \ldots, \lambda_m$ be all elements of $\Sigma$ which can be written in the form $\lambda_i = \sum_{i=1}^{r}r_i e_i$ with $0 \leq r_i < 1.$ Then, for any $1 \leq i \leq j \leq m,$ we can write $\lambda_i + \lambda_j$ in the above canonical form, yielding a relation in degree at most $\deg \lambda_i + \deg \lambda_j \leq 2 \cdot \left( \sum_{i=1}^{r}d_r -1 \right).$ Furthermore, these relations generate all relations, as one can apply a sequence of these relations to put any $\sigma \in \Sigma$ into canonical form $\sigma = \lambda + \sum_{i=1}^{r}w_i e_i$ as in Proposition ~\ref{prop:cone-generation}.
\end{proof}


\section{Canonical Rings on Projective Space}
\label{sec:proj}
In this section, we bound the degrees of generators and relations for divisors on $\bp^m$, for all $m \geq 1$. The $\bp^1$ case was covered in \cite{dorney:canonical}.

\ssec{Preliminaries on Projective Space}

For the remainder of this section, we shall fix $m \geq 1$ and choose an isomorphism $\bp^m \cong \proj V$ so that $x_0,\ldots, x_m$ form a basis for $V$.  We will think of $x_i$ as a rational section of $H^0(\bp^m, \sco_{\bp^m})$ of degree 1.

Write
\begin{align*}
	D = \sum_{i=0}^{n}\alpha_i D_i.
\end{align*}
where $n \in \bz$, $\alpha_i \in \bq$, and $\deg D_i = \bida_i$. Let $D_i$ be Cartier divisors such that $D_i = V(f_i)$.
We shall further assume that (after reordering) $f_0,\ldots, f_{m}$ form a basis for degree one rational functions on $\bp^m$.

\begin{prop}
\label{prop:pm-span-and-basis}
The functions $u^d \cdot \prod_{i=1}^n f_i^{c_i}$ satisfying
\begin{align}
\label{align:pm-span}
\sum_{i=1}^{n} c_i \cdot \bida_i = 0 && \text{ and } &&c_i \geq \lfloor \alpha_i d\rfloor	
\end{align}

\noindent
form a spanning set for $H^0(\bp^m, dD)$ over $k$. Furthermore, functions 
of the form $u^d \cdot \prod_{i=1}^n f_i^{c_i}$ satisfying
\begin{align}
\label{align:pm-basis}
c_i = -\lfloor d\alpha_i \rfloor && \text{ for } && i > m
\end{align}

\noindent
form a basis for $H^0(\bp^m, dD)$ over $k$.
\end{prop}

\begin{proof}
By definition of $H^0(\bp^m,dD)$, functions satisfying conditions 
~\eqref{align:pm-span} lie in $H^0(\bp^m,dD)$. To complete the proof, it suffices to check functions satisfying conditions ~\eqref{align:pm-basis} form a basis of $H^0(\bp^m,dD)$. Note that there $\binom{m+ \lfloor dD \rfloor }{m}$ functions satisfying condition ~\eqref{align:pm-basis}. However we know $h^0(\bp^m,dD) = \binom{m+ \lfloor dD \rfloor }{m},$ so it suffices to show that those functions satisfying condition ~\eqref{align:pm-basis} are independent. This follows from the assumption that $f_0,\ldots f_n$ form a basis of degree 1 rational functions, and so degree $\lfloor d \deg D \rfloor $ monomials in $f_0,\ldots, f_n$ form a basis of degree $\lfloor d \deg D \rfloor $ rational functions.
\end{proof}

\ssec{Effective Divisors on Projective Space}
\label{ssec:proj-one-point}

In this subsection, we restrict attention to the case of effective
fractional divisors $D \in \bq \otimes \di \bp^m$. We completely
characterize the canonical rings when $D$ consists of one
hypersurface and give strong bounds on generator and relation
degrees for general effective divisors $D$.

\begin{convention}
Let $[x_0, \ldots, x_m]$ be coordinates for $\bp^m$. Let
$\vec{v} = (v_0, \ldots, v_m) \in \br^{m + 1}$. Then write
\[
	x^{\vec{v}} := \prod_{i = 0}^{m} x_i^{v_i}
\]
\end{convention}

\begin{defn}
\label{defn:vec-sum}
For $\vec{v} \in \br^n$, denote $\deg \vec{v} := \sum_{i = 0}
^n v_i$.
Let 
\begin{align*}
	\mss_i := \left \{\vec{v} \in \bz_{\geq 0}^{m + 1} \; : \;
\deg \vec v = c_i \right\}.	
\end{align*}
\end{defn}

\begin{defn}
\label{defn:vec-order}
Let $\vec{v}, \vec{w} \in \br^n$. Let $i \in \{1,\ldots, n\}$
be the biggest index such that $v_i$ is nonzero
and $j \in \{1,\ldots, n\}$ be the smallest index such that $w_j$ is
nonzero. Define $\vec{v} \prec \vec{w}$ if $i \leq j$.
\end{defn}

\begin{thm}
\label{thm:proj-one-point}
Let $\bp^m \cong \proj \bk [x_0, \ldots, x_m].$ Let $D = \alpha H$, with $\alpha = \frac{p}{q} \in \bq$
and $H := V(x_k)$ a hyperplane of $\bp^m$.
Let
\[
	0 = \frac{c_0}{d_0} <
	\frac{c_1}{d_1} < \ldots < \frac{c_r}{d_r} = \frac{p}{q}
\]

\noindent
be the best lower approximations of $\alpha$. Then, the
canonical ring
\[
	R_D := \bigoplus_{d \geq 0} H^0(\bp^m, \lfloor dD \rfloor)
\]

\noindent
has a minimal presentation consisting of the $\sum_{i = 0}^{r}
{{m + c_i - 1} \choose {c_i}}$ generators $F_i^{\vec{v}} := \frac{u^{d_i}
x^{\vec{v}}}{x_k^{c_i}}$ where $1 \leq i \leq r$, $\vec{v} \in \bz_{\geq 0}^{m + 1}$
with $\deg \vec v = c_i$. Furthermore, it has
relations of the following two types:
\begin{enumerate}
	\item For each $(i, j)$ with $j \geq i + 2$ and each $\vec{v} \in \mss_i,
\vec{w} \in \mss_j$, there is a relation of the form either
\begin{flalign*}
	&G_{i, j}^{\vec{v}, \vec{w}} = F_i^{\vec{v}} F_j^{\vec{w}}
	- \prod_{\vec{y} \in \mss_{h_{i, j}}} (F_{h_{i, j}}^{\vec{y}})
	^{g_{\vec{y}}} &(i < h_{i, j} < j) \\
	\text{or} \\
	&G_{i, j}^{\vec{v}, \vec{w}} = F_i^{\vec{v}} F_j^{\vec{w}}
	- \prod_{\vec{y} \in \mss_{h_{i, j}}} (F_{h_{i, j}}^{\vec{y}})^{g_{\vec{y}}}
	\cdot \prod_{ \vec{z} \in
	\mss_{h_{i, j} + 1}} (F_{k_{i, j} + 1}^{\vec{z}})^{g'_{\vec{z}}}
	&(i < h_{i, j} < h_{i, j} + 1 < j).
\end{flalign*}
	\item For each $(i, j)$ with
$j = i$ or $j = i + 1$ and each $\vec{v} \in \mss_i, \vec{w} \in
\mss_j$ with $\vec{v} \not\prec \vec{w}$ (see Definition
~\ref{defn:vec-order}) there is a relation of the form
\begin{align*}
	&L_{i, j}^{\vec{v}, \vec{w}} = F_i^{\vec{v}} F_j^{\vec{w}}
	- F_i^{\vec{y}} F_j^{\vec{z}} \\
\end{align*}

\end{enumerate}

\noindent
where $\vec{y}$ and $\vec{z}$ are the unique
vectors in $\mss_i$ and $\mss_j$, respectively, such that $\vec{y}
+ \vec{z} = \vec{v} + \vec{w}$ and $\vec{y} \prec \vec{z}$.
\end{thm}
\todo{Write idea of proof}
\begin{proof}
We will first check that $F^{\vec v}_i$ determine a system of minimal generators of $R_D$. We know $\left\{F_{1}^{\vec{v}} : \deg \vec{v} = \left\lfloor
\frac{p}{q} \right\rfloor \right\}$ are minimal generators in
degree $1$. Now let $i > 1$. Suppose $F_i^{\vec{v}}$ were not
minimal for some $\vec{v} \in \mss_i$. Then $F_i^{\vec{v}} =
F_{s}^{\vec{y}} F_{t}^{\vec{z}}$ for some $s, t \in \bn$ and
$\vec{y}, \vec{z} \in \bz_{ \geq 0}^{m + 1}$. In particular, we see
that 
\begin{align*}
	&d_i = s + t \\
	&c_i = \deg \vec{v} = \deg \vec{y} + \deg \vec{z}
\end{align*}

\noindent
so $\frac{c_i}{d_i}$ is the mediant of the rational numbers
$\frac{\deg \vec{y}}{s}, \frac{\deg \vec{z}}{t}$ which both
do not exceed $\alpha$. One of $\frac{\deg \vec{y}}{s},
\frac{\deg \vec{z}}{t}$ must be at least $\frac{c_i}{d_i}$
and $s$ and $t$ are both less than $d_i$, which contradicts the assumption that $\frac{c_i}{d_i}$ is a best lower approximation of $\alpha$
(see Definition ~\ref{defn:lower-approximation}).

Next we show that $F^{\vec v}_i$ generate all of $R_D$. In the case $m = 1$, O'Dorney \cite[Theorem 6]
{dorney:canonical} demonstrates that each lattice point $(\beta, \gamma) \in
\bz_{\geq 0}^2$ with $\gamma \leq \beta \alpha$ lies in the $\bz_{\geq 0}$ span of $(d_h, c_h)$ and $(d_{h + 1}, c_{h + 1})$ for
some $h \in \{0, \ldots, r\}$. Let $(\beta, \gamma) = \kappa_1
(d_h, c_h) + \kappa_2 (d_{h + 1}, c_{h + 1})$ for $\kappa_1, \kappa_2 \in
\bz_{\geq 0}$. Any element $\frac{u^{\beta}
x^{\vec{v}}} {x_k^{ \gamma}} \in R_D$ is expressible as
\begin{align*}
	\frac{u^{\beta} x^{\vec{v}}} {x_k^{\gamma}} = \left(\frac{u^{d_h}}
	{x_k^{c_h}}\right)^{\kappa_1} \left(\frac{u^{d_{h + 1}}}
	{x_k^{c_{h + 1}}}\right)^{\kappa_2} x^{\vec{v}}.
\end{align*}

\noindent
We can then write $\vec{v}  = \sum_{\lambda=1}^{\kappa_1}\vec{w}_{(\lambda)} + \sum_{\eta=1}^{\kappa_2} \vec z_{(\eta)}$ with $w_{(\lambda)} \in \mss_h$ and $z_{(\eta)} \in \mss_{h+1}$ to give a decomposition
\begin{align}
\label{eqn:one-point-canonical-form}
	\frac{u^{\beta} x^{\vec{v}}} {x_k^{\gamma}}	= \prod_{\lambda = 1}
	^{\kappa_1} \frac{u^{d_h} x^{\vec{w}_{(\lambda)}}} {x_k^{c_h}}
	\prod_{\eta = 1}^{\kappa_2} \frac{u^{d_{h + 1}} x^{\vec{z}_{(\eta)}}}
	{x_k^{c_{h + 1}}}
\end{align}
\noindent
consisting of products of generators $F_h^{\vec{y}_{(\lambda)}}$
and $F_{h + 1}^{\vec{z}_{(\eta)}}$ which are in the form
prescribed in the theorem statement. Since we wrote an arbitrary monomial $\frac{u^{\beta} x^{\vec{v}}}{x_k^\gamma} \in R_D$ as a product of generators, this shows that $F_i^{\vec v}$ minimally generate $R_D$.

Next, we show that the relations given in the statement of the theorem generate all relations. 
In particular, if $j \geq i + 2$ then $F_i^{\vec{v}} F_j^{\vec{w}}$
has a decomposition of the form ~\eqref{eqn:one-point-canonical-form} where $h$ depends on $i, j$, so we notate $h = h_{i, j} \in \{1, \ldots, r\}$. We also have
that $i \leq h_{i, j} < j$ since $(d_i + d_j, c_i + c_j)$ is in the
$\bz_{\geq 0}$-span of $(d_{h_{i, j}}, c_{h_{i, j}})$ and
$(d_{h_{i, j} + 1}, c_{h_{i, j} + 1})$. Furthermore, we have that
$h_{i, j} \neq i, j-1$ as can be shown in the same manner as the analogous statement in O'Dorney
\cite[Theorem 6]{dorney:canonical}. This gives the relations
$G_{i, j}^{\vec{v}, \vec{w}}$.

The application of $G_{i, j}^{\vec{v}, \vec{w}}$ can be used
to transform any monomial in the $F_i^{\vec{v}}$'s involving
indices that differ by more than $1$ to a monomial in the $F_i
^{\vec{v}}$'s involving indices that differ by at most $1$.

We also have relations involving generators in consecutive
indices. Suppose $F_i^{\vec{v}}$ and $F_j^{\vec{w}}$ are
generators with $j = i$ or $j = i + 1$ and $\vec{v} \in
\mss_i, \vec{w} \in \mss_j$ with $\vec{v} \not\prec \vec{w}$.
Let $\vec{y}$ and $\vec{z}$ be the unique vectors in $\mss_i$ and
$\mss_j$, respectively, such that $\vec{y} + \vec{z} = \vec{v} +
\vec{w}$ and $\vec{y} \prec \vec{z}$ (i.e.~ the nonzero indices
of $\vec{y}$ followed by those of $\vec{z}$ give an increasing
sequence). Then we see that
\begin{align*}
	F_i^{\vec{v}} F_j^{\vec{w}} = x^{\vec{v} + \vec{w}}
	\left(\frac{u^{d_i}}{x_k^{c_i}}\right)
	\left(\frac{u^{d_j}}{x_k^{c_j}}\right)
	= x^{\vec{y} + \vec{z}}
	\left(\frac{u^{d_i}}{x_k^{c_i}}\right)
	\left(\frac{u^{d_j}}{x_k^{c_j}}\right)
	= F_i^{\vec{y}} F_j^{\vec{z}},
\end{align*}

\noindent 
which give the relations $L_{i, j}^{\vec{v}, \vec{w}}$.

Now, we may apply the relations $L_{i, j}^{\vec{v}, \vec{w}}$
to any monomial in the $F_i^{\vec{v}}$'s involving
indices that differ by at most $1$ to produce the canonical form
\begin{align*}
	(F_i^{\vec{v}_{(1)}})^{g_{\vec{v}_{(1)}}} \cdots
	(F_i^{\vec{v}_{(\kappa_1)}})^{g_{\vec{v}_{(\kappa_1)}}}
	(F_{i + 1}^{\vec{w}_{(1)}})^{g_{\vec{w}_{(1)}}} \cdots
	(F_{i + 1}^{\vec{w}_{(\kappa_2)}})^{g_{\vec{w}_{(\kappa_2)}}}
\end{align*}

\noindent
where $\vec{v}_{(1)} \prec \vec{v}_{(2)} \prec \ldots \prec
\vec{v}_{(\kappa_1)} \prec \vec{w}_{(1)} \prec \ldots \prec
\vec{w}_{(\kappa_2)}$. Consequently, the relations of form
$G_{i, j}^{\vec{v}, \vec{w}}$ and of form $L_{i, j}^{\vec{v},
\vec{w}}$ generate all the relations among the $F_{i}^{\vec{v},
\vec{w}}$.
\end{proof}

\begin{rem}
\label{rem:proj-two-points}
Note that we cannot extend this result to the case when
$D$ is supported at two hypersurfaces with arbitrary rational
coefficients in the same manner that O'Dorney does for the
$\bp^1$ case \cite[Section 4]{dorney:canonical}. As will be shown
in Example ~\ref{eg:hyperplane}, the degrees of generation of the
canonical ring of a general $\bq$-divisor supported on two
hyperplanes cannot be bounded so tightly. The two-point
$\bp^1$ result leverages particular facts about the structure of
$\bp^1$. \todo{Possibly say what facts?}
\end{rem}

\begin{rem}
\label{rem:proj-one-point-ind}
The method of proof for Theorem ~\ref{thm:proj-one-point}
can be immediately generalized to work inductively. If $D = D' + \alpha H$
for some hyperplane $H$, then the proof of Theorem ~\ref{thm:proj-one-point}, mutatis mutandis, extends to show
the canonical ring $R_D$ is generated over $R_D'$ by the same
generators and with the same relations given in Theorem
~\ref{thm:proj-one-point}.
\end{rem}
\todo{Perhaps we should consider restating the above theorem in this form, and then say for simplicity we prove it when the base is 0?}
\begin{thm}
\label{thm:proj-effective}
Let $[x_0, \ldots, x_m]$ be a projective coordinate system for
$\bp^m$. Let $D = \sum_{i = 1}^{n} \alpha_i D_i \in \bq \otimes \bp^m$, with $\alpha_i =
\frac{p_i}{q_i} \in \bq$.

Then the canonical ring
$R_D$
is generated in degrees up to $\max_{1 \leq i \leq n}{q_i}$ with
relations generated in degrees up to $2 \max_{1 \leq i \leq n}{q_i}$.
\end{thm}

\begin{proof}
We derive the result from the following inductive argument.
If $n = 0$, i.e.~ $D = 0$, then we are done.
Now, we look to inductively add hypersurfaces.
If $D' = \sum_{i = 0}^{n} \alpha_i D_i$ is some $\bq$-divisor
on $\bp^m$, then we look at $D = D' + C$ where $C$ is a
hypersurface of degree $\pdeg$. \todo{Aaron: rewrite this to make sense} Now use the Veronese mapping
to view $C$ as a hypersurface on $\bp^{\binom{{m + \pdeg}}{ \pdeg} -1 }$
and the image of $C$ as the intersection of a Veronese variety
with a hyperplane. Then we can reduce inductively adding hypersurfaces
to the case of adding hyperplanes. However, this is accomplished
by the method used in Theorem ~\ref{thm:proj-one-point} in an
inductive manner as mentioned in Remark ~\ref{rem:proj-one-point-ind}.
\end{proof}


\ssec{Bounds for Arbitrary Divisors on Projective Space}


We now offer bounds on generators and relations of $R_D$ for a general divisor $D$ on $\bp^m$.

Let $D = \sum_{i=1}^n \alpha_i D_i \in \bq \otimes \di \bp^m$ and let $f_i$ be homogeneous polynomials in $x_0, \ldots, x_m$ such that $D_i = V(f_i)$ $\deg(f_i) = a_i$ for all $i$. 

For the remainder of this section, we shall make the additional assumption that 
\begin{align}
\label{eqn:pm-basis-assumption}
	f_1, \ldots, f_{m+1} \text{ form a } \bk \text{ basis of all degree 1 functions.}
\end{align}

Our first aim in this section is to prove Theorem ~\ref{thm:proj-generators-relations} bounding the number of generators and relations of $R_D$. Before doing so, we justify the assumption ~\eqref{eqn:pm-basis-assumption} through a several convincing examples.


\begin{example}
\label{eg:hyperplane}
Suppose $D = \frac{1}{2}H_0 - \frac{1}{3}H_1$ where $H_0 = V(x_0),
H_1 = V(x_1)$ are two coordinate hyperplanes in $\bp^2$. Then, $R_D$
has generators in degree $2$ and $3$ which can be written as $u^2
\frac{x_1}{x_0}, u^3 \frac{x_1}{x_0}.$ In fact, for all degrees
less than $5$ the elements of $R_D$ can all be expressed as
rational functions in $x_0, x_1$. However, in degree $6$, there is
the rational function $\frac{x_1^2 x_2}{x_0^3}$. Since this
involves $x_2$, it must be a generator.

This example generalizes slightly to any divisor of the form $D =
\frac{1}{k}H_0 - \frac{1}{k+1}H_1 \in \di \bp^2 \otimes \bq,$ with
$k \in \bn$, showing that there will always exist a generator in
degree $a(a + 1)$.

This example further generalizes to the following situation:
Suppose $D = \sum_{i=1}^{n} \frac{\num_i}{\den_i}D_i \in \di \bp^m
\otimes \bq,$ where $\deg D_i = \bida_i$ and $\deg D = \frac{1}{\lcm
_{1 \leq i \leq n}(\den_i \cdot \bida_i)}$. Then, if $D_i = V(f_i)$
where all $f_i$ can be written as a polynomial function in $x_0,
\ldots, x_{m-1},$ we have $R_D$ always has a generator in degree
$\lcm(\den_i \cdot \bida_i)$.
\end{example}

\begin{rem}
\label{rem:ghost-motivation}
Example ~\ref{eg:hyperplane} shows that the naive generalization of
~\cite[Theorem 8]{dorney:canonical} of generation in degree at most
$\sum_{i=1}^{n}\ell_i$ cannot possibly hold. The reason for this is
that the divisors may ``be contained in a hyperplane,'' i.e.~ be
expressible as functions in $m$ of the $m+1$ variables on $\bp^m$.
This problem can easily be circumvented by adding in ``ghost
points.'' That is, we may add divisors of the form $0 \cdot H_i$ to
$D$, and reorder so that if $D = \sum_{i=1}^{n}\alpha_i V(f_i)$,
then $f_0, \ldots, f_n$.

In Example ~\ref{eg:radical}, we show that it is still, in general,
necessary to add ghost points, even when the irreducible components
of a divisor are not all contained in a hyperplane.
\end{rem}

\begin{example}
\label{eg:radical}
Consider $\frac{-1}{5}V(x_0^2 + x_1^2 + x_2^2) + \frac{1}{7}V(x_0^2 + x_1^2 + x_3^2) + \frac{1}{17}V(x_0^2 + x_2^2 + x_3^2) - \frac{1}{596}V(x_1^2 + x_2^2 + x_3^2)$. In degree $355216 = 5 \cdot 7 \cdot 17 \cdot 596$, this has dimension $10$. 
\todo{Clarify what this example means by the lattice approach}
However, we possibly use the lattice approach, as all lower degrees only have dimension 1. In fact, we may construct similar examples using any set of degree $n$ divisor with very small degree.
\end{example}
\todo{construct class of additional examples}

Having justified the necessity of adding ghost points, we proceed to bound the number of generators and relations.

\begin{lem} \label{lem:proj-generators}
$R_D$ is generated in degrees up to $\sum_{i=1}^n \ell_i a_i.$
% and relations in degree at most  .\todo{insert numbers}
\end{lem}
\begin{proof}
Let 
\begin{align}\label{eqn:Sigma-def}
	\Sigma = \left \{(d, c_1, \ldots, c_n) \in \bz^{n+1} : c_i \geq - d \alpha_i, \; 1 \leq i \leq n, \text{ and } \sum_{i=1}^{n} \ell_i \bida_i = 0 \right \}.
\end{align}
\todo{Possibly take this definition out of the proof, and move it into the setup, since we want to reference it later.}
We note that $\Sigma$ has extremal rays given by the lattice points 
\begin{equation}\label{defn:e-i-proj}
	e_i = \left(\ell_i \bida_i, - \alpha_1 \ell_i \bida_i, \ldots -\alpha_{i-1} \ell_i \bida_i, \ell_i \sum_{j\ne i} \alpha_j \bida_j, -\alpha_{i+1} \ell_i \bida_i, \ldots, -\alpha_n, \ell_i \bida_i \right)
\end{equation}
for each $i\in \{1, \ldots n\}$.
Therefore, applying Proposition ~\ref{prop:cone-generation}, we see $R_D$ is generated in degrees less than
\[
	\sum_{i=1}^n \ell_i a_i.
\]
\end{proof}

\begin{rem}
\label{rem:pm-extremal-rays}
Note for future reference that the extremal rays of the cone $\Sigma$, as defined in the proof of Lemma ~\ref{lem:proj-generators} correspond to the elements
\begin{equation}
\label{eqn:epsilon-def-proj}
	\epsilon_i = \prod_{j\ne i} (f_j)^{-\alpha_j \ell_i a_i} {f_i}^{\ell_i \sum_{j\ne i} \alpha_j a_j}\in R_D.
\end{equation}
\end{rem}



Let $x_1, \ldots x_r$ be the generators in degrees at most $\sum_{i=1}^n \ell_i \bida_i$ (given by Lemma ~\ref{lem:proj-generators}), and let $\phi$ be the surjection $k[x_1, \ldots x_r] \to R_D$.  We can factor $\phi$ through the semigroup ring 
\[
	\bk[\Sigma] =  \langle u^d z_1^{c_1} \cdots z_n^{c_n} : c_i \in \mathbb{Z}, \; c_i \geq -d \alpha_i, \mbox{ and }\sum_{i=1}^{n} \bida_i c_i = \sum_{i=1}^{n} \bidb_i c_i \rangle. 
\]
by
\[
\begin{tikzcd}
\bk[x_1,\ldots, x_r] \ar {r}{\chi} & \bk[\Sigma] \ar {r}{\psi} & R_D \\
x_i \ar r & u^{d_i}z_1^{c_{i1}} \cdots z_n^{c_{in}} \ar r & u^{d_i}f_1^{c_{i1}} \cdots f_n^{c_{in}}.
\end{tikzcd}
\]

By Lemma ~\ref{lem:bound-ker-chi}, we can bound the degree of generation of $\ker(\chi)$ and we next calculate the degree of generation of $\ker \psi$:

\begin{lem}
\label{lem:proj-relations-psi}
$\ker(\psi)$ is generated in degrees up to
\begin{align}
\label{eqn:proj-relation-degree}
	\frac{\max_{1\le i \le n}(\bida_i)}{\deg(D)} +  \sum_{i=1}^n \ell_i a_i.
\end{align}

\end{lem}

\begin{proof}
We first claim there exist $\beta_1, \ldots, \beta_n \in k[\Sigma]$
such that $\ker \psi$ is generated by
\begin{equation}
\label{eqn:relations-psi-proj}
	(z_i - \beta_i)\prod_{j=1}^n {z_j}^{c_{j}}
\end{equation}

\noindent
for all $d \in \mathbb{N}$ and $c_i \ge -\alpha_i d$ satisfying
$\bida_i + \sum_{j = 1}^n \bida_j c_j = 0$.

We define the $\beta_i$ above so that $\beta_i$ is the expansion of
$f_i$ when viewed as a polynomial in $k[f_1,\ldots, f_{m + 1}]$.
Furthermore, the relations given in Equation
~\eqref{eqn:proj-relation-degree} generate all relations, since they
allow us to reduce any $ \prod_{j = 1}^n f_j^{r_i}$ to a canonical
form, with $r_i = -\lfloor  d \alpha_i\rfloor$ whenever $i  > m + 1$.

For the remainder of the proof, fix $i \in \{1,\ldots, n\}$. To
complete the proof, it suffices to bound the degree of generation
of the relations of the form $(z_i - \beta_i) \prod_{j = 1}^n
z_j^{c_j}$, by Equation ~\eqref{eqn:proj-relation-degree}. We first
associate $(z_i - \beta_i)\prod_{j=1}^n z_j^{c_j}$ with $(d, c_1,
\ldots, c_n) \in \Sigma,$ as defined in Equation
~\eqref{eqn:Sigma-def}. Let $\Sigma_i \subseteq \mathbb{Z}^{n + 1}$
be the set of points of the form $(d, r_1, \ldots, r_n)$ satisfying
$r_j \ge -d \alpha_j$ for all $j$ and $\sum_{j=1}^n r_j a_j = -a_i$.
Let
\begin{align*}
	\delta_i = \left(\frac{a_i}{\deg(D)}, -\frac{\alpha_1 a_1}{\deg(D)},
	\ldots, - \frac{\alpha_n a_n}{\deg(D)} \right).
\end{align*}

\noindent
Then we see $\Sigma_i \subseteq \delta_i + \newspan_{\br_{\geq 0}}
(e_1, \ldots, e_n)$ with $e_i$ as defined in Equation
~\ref{defn:e-i-proj}. Therefore, we can write any element of
$\Sigma_i$ as
\[
	\delta_i + \sum_{j=1}^n r_j e_j
\]
where $r_j \in \mathbb{R}$ for each $j$.

Whenever some $r_j \ge 1$, we can generate the relation $(z_i -
\beta_i)\prod_{j=1}^n z_j^{c_j} = \epsilon_j \cdot h,$ where
$\epsilon_j$ was defined in Equation ~\ref{eqn:epsilon-def-proj} and
$h \in \Sigma_i$. Therefore, $\ker(\psi)$ is generated in degrees
less than
\[
	\frac{\bida_i}{\deg(D)} + \sum_{i=1}^n \ell_i a_i.
\]
\end{proof}

Combining the above results, we obtain the following bound on the
degrees of generators and relations of $R_D$.

\begin{thm}
\label{thm:proj-generators-relations}
Let $D = \sum_{i=1}^n \alpha_i D_i \in \bq \otimes \di \bp^m$. Let
$f_i$ be homogeneous polynomials in $x_0, \ldots, x_m$ such that
$D_i = V(f_i)$ for all $i$; suppose $\deg(f_i) = a_i$. 
Then $R_D$ is generated in degree at most 
\[
	\sum_{i=1}^n \ell_i a_i
\]

\noindent
with relations generated in degree at most
\[
	\max \left(2 \sum_{i=1}^n \ell_i a_i, \frac{\max_{1\le i \le n}
	(	\bida_i)}{\deg(D)} + \sum_{i=1}^n \ell_i a_i \right).
\]
\end{thm}

\begin{proof}
The bound on degree of generation is precisely the content of Lemma 
~\ref{lem:proj-generators}. It only remains to bound the degree of 
relations.

By ~\ref{lem:bound-ker-chi}, $\ker(\chi)$ is generated in degrees 
up to $2\sum_{i=1}^n \ell_i a_i$ and by Lemma
~\ref{lem:proj-relations-psi}, $\ker(\psi),$ is generated in
degrees up to $\frac{\bida_i}{\deg(D)} + \sum_{i=1}^n \ell_i a_i$. 
Consequently, Lemma ~\ref{lem:composite-map} implies that $\ker(\phi)$
is generated in degrees less than
\[
	\max \left(2 \sum_{i=1}^n \ell_i a_i, \frac{\max_{1\le i \le n}
	(\bida_i)}{\deg(D)} + \sum_{i=1}^n \ell_i a_i \right).
\]
\end{proof}

\section{Canonical Rings on Hirzebruch surfaces}
\label{sec:hirz}
In this section, the main aim is to prove Theorem
~\ref{thm:hirz-generators-relations}, bounding the degree of
generators and relations of the canonical ring of a $\bq$-divisor
on any Hirzebruch surface.

\todo{Aaron: Give some definitions to make clear what the structure of divisors are}
\todo{Aaron: clarify that $R_D$ is assumed to have both positive bidegrees}
\todo{Aaron: Have $\ell_{i,j}$ been defined?}
Let $D=\sum_{i=1}^n \alpha_i D_i \in \mathbb{Q} \otimes \di(\hirz_
m).$ such
that $D_1, D_2, D_3,$ and $D_4$ are distinct divisors with bi-degrees $(1,0)
, (1,0), (0,1)$, and $(0,1)$ respectively. Let $D_i = V(f_i)$ where
$f_i \in \sco(\bida_i, \bidb_i)$.  
Then, $f_1$ and $f_2$ are independent linear
polynomials in $x_0$ and $x_1$ and $f_3$ and $f_4$ are independent
linear polynomials in $y_0$ and $y_1$. \todo{Talk a little about Hirzebruch surfaces before this paragraph; especially, describe what it's sections look like.}

\begin{defn}
Define 
\begin{equation}
\label{eqn:define-T=(D)}
	\Te(D) = \left\{i \in \{1, \ldots, n\}: \bida_i \sum_{k=1}^n \bidb_k 
\alpha_k = \bidb_i \sum_{k=1}^n \bida_k \alpha_k \right\},
\end{equation}

\begin{equation}
\label{eqn:define-T+(D)}
	\Tp(D) = \left\{ i \in \{1, \ldots, n\}:  \vphantom{\sum_{k = 1}^n} 
	\bida_i \sum_{k = 1}^n \alpha_k \bidb_k > \bidb_i \sum_{k = 1}^n \alpha_k \bida_k 
\right\},
\end{equation}

\noindent
and
\begin{equation}
\label{eqn:define-T-(D)}
	\Tm(D) = \left\{ j \in \{1, \ldots, n\}: \bida_j \sum_{k = 1}^n \alpha_k
	\bidb_k < \bidb_j \sum_{k=1}^n \alpha_k \bida_k \right\}.
\end{equation}
\end{defn}

\begin{lem}
\label{lem:hirz-generators}
$R_D$ is generated in degree at most
\[
	\sum_{i \in \Te(D)} \gcd(\bida_i, \bidb_i) \ell_i + \sum_{\substack
	{i \in \Tp(D) \\ j\in \Tm(D)}} (\bida_i \bidb_j - \bida_j \bidb_i)
	\ell	_{i,j}.
\]
\end{lem}

\begin{proof}
%Let $(\bida_i, \bidb_i)$ be the bi-degrees of $f_i$. 
Suppose $\gamma
\in (R_D)_d$. \todo{Aaron: Why is $\gamma$ defined? It doesn't seem to be used anywhere.}
 Then 
\[
	g = \sum_{i = 1}^n {f_i}^{c_i}
\]

\noindent
where each $c_i \ge - \alpha_i c_i$, $\sum_{i=1}^n c_i \bida_i = 0$,
and $\sum_{i=1}^n c_i \bidb_i = 0$. We can any view $\sigma$ inside the
lattice 
\begin{align*}
	\Sigma = \{ (d, c_1, \ldots, c_n) \in \bz_{\geq 0}^{n+1} : c_i \ge - d \alpha_i \text{ for all }i \text{ and }\sum_{i=1}^n c_i \bida_i = \sum_{i = 1}^n c_i \bidb_i = 0 \}.	
\end{align*}

By Proposition ~\ref{prop:cone-generation}, in order to determining a spanning set for $(R_D)_d,$ it suffices to find the extremal rays of $\Sigma.$
To do this, we extend the method of O'Dorney \cite[Theorem 8]{dorney:canonical}. 
We first consider the sub-cone $\Sigma_1 \subset \Sigma$
given by
\begin{align*}
	\Sigma_1 = \{ (d, c_1, \ldots, c_n) \in \bz_{\geq 0}^{n+1} : c_i \ge - d \alpha_i \text{ for all }i \text{ and }\sum_{i=1}^n c_i (\bida_i+\bidb_i) = 0 \}.	
\end{align*}
Then, $\Sigma_1$
has extremal rays given by
 
\[
	e_i := (1, -\alpha_1, \ldots, -\alpha_{i-1}, \frac{\sum_{j \ne i}
	\alpha_j (\bida_j + \bidb_j)}{\bida_i + \bidb_i}, -\alpha_{i + 1},
	\ldots, -\alpha_n).
\]
for $1 \leq i \leq n$.

Let $\Delta^{n-1}$ be the simplex with vertices $e_1, \ldots e_n$.
We can intersect $\Delta^{n-1}$ with the hyperplane $H$ given by
$\sum_{i=1}^n \bida_i X_i = 0$  to get a subspace $S \subseteq \Delta
^{n-1}$. The extremal rays of $S$ must lie on the edges of $\Delta
^{n - 1}$ \todo{Peter: Do we need to cite/argue this, or is it clear?}.
Thus, the extremal rays of $S$ are given by the $e_i$'s
contained in $H$ and the intersection points $e_{i, j}$ of $H$
and the edges $\overline{e_i e_j}$ of the simplex between
distinct $e_i, e_j$ not contained in $H$.
\todo{more detail above}

We can calculate elements of $R_D$ whose
corresponding lattice points lie on these extremal rays:

For $i \in \Te(D)$, define \todo{below section needs to be simplified, maybe
by introducing new variables}
\[
	\epsilon_i := (\prod_{k \ne i} f_k^{-\alpha_k \ell_i \gcd(\bida_i, \bidb_i)})
	(f_i^{\ell_i \frac{\gcd(\bida_i, \bidb_i)}{\bida_i + \bidb_i}\sum_{k \ne i}
	\alpha_k (\bida_k + \bidb_k)}) \in H^0(\ell_i \gcd(\bida_i, \bidb_i) D).
\]

\noindent
For $i \in \Tp(D)$ and $j \in \Tm(D)$, define
\[
	\epsilon_{i, j} := \left(\prod_{k \ne i, j} f_k^{-\alpha_k \ell_{i,j} (\bida_i \bidb_
	j - \bida_j \bidb_i)}) {f_i}^{s_1} {f_j}^{s_2} \in H^0(\ell_{i, j}
	(\bida_i \bidb_j - \bida_j \bidb_i) D \right)
\]

\noindent
where
\[
	s_1 = \bidb_j \sum_{k \ne i,j} \alpha_k \bida_k - \bida_j \sum_{k \ne i, j}
	\alpha_k \bidb_k = b_j (\alpha_i a_i + \alpha_j a_j) - a_j (\alpha_i b_i + \alpha_j b_j)
\]

\noindent
and
\[
	s_2 = -\bidb_i \sum_{k \ne i, j} \alpha_k \bida_k + \bida_i \sum_{k \ne i, j}
	\alpha_k \bidb_k = -b_i (\alpha_i a_i + \alpha_j a_j) + a_i (\alpha_i b_i + \alpha_j b_j).
\]
\todo{Aaron: Why are $\epsilon_{i,j}$ defined here instead of $e_{i,j}$. We don't use the $\epsilon$ here, and we do use the $e$ here.}

The set of $e_i$ and $e_{i,j}$ given above satisfies the conditions of Proposition ~\ref{prop:cone-generation}, $R_D$ is generated in 
degrees less than the sum of the degrees of the $e_i$ and $e_{i,j}$, which is
\[
	\sigma = \sum_{i\in \Te(D)} \gcd(\bida_i, \bidb_i)\ell_i + \sum_{\substack{
	i \in \Tp(D) \\	j \in \Tm(D)}} (\bida_i \bidb_j- \bida_j \bidb_i)\ell_{i,j}.
\]
\todo{Aaron: Do you use $\sigma$ later? We are usually reserving $\sigma$ for a lattice point. Perhaps something like $\delta$ would be more appropriate if you want to refer to it later. Also, you should number if if you want to refer to it later. In general, it's bad style to define things in the proofs. This could have been defined prior to the theorem statement as well, which would simplify the statement. We should also then clarify that $\sigma$ depends on $D$.}

\end{proof}

Let $x_1, \ldots x_r$ be the generators of $R_D$ in degrees less than $\sigma$ (as given by Lemma ~\ref{lem:hirz-generators}), and let $\phi$ be the surjection $k[x_1, \ldots x_r] \to R_D$.  We can factor $\phi$ through the semigroup ring 
\[
	\bk[\Sigma] = \left \langle u^d z_1^{c_1} \cdots z_n^{c_n} : c_i \geq -d
	\alpha_i, \sum_{i=1}^{n} \bida_i c_i = \sum_{i=1}^{n} \bidb_i c_i
	\right \rangle.
\]

\noindent
by
\[
\begin{tikzcd}
	\bk[x_1,\ldots, x_r] \ar {r}{\chi} & \bk[\Sigma] \ar {r}{\psi} & R_D \\
	x_i \ar r & u^{d_i}z_1^{c_{i1}} \cdots z_n^{c_{in}} \ar r & u^{d_i}f_1^{c_{i1}} \cdots f_n^{c_{in}}.
\end{tikzcd}
\]

By Lemma ~\ref{lem:bound-ker-chi}, we can bound the degree of generation of $\ker(\chi)$ below
$2 \sigma$.
Finally, we calculate the degree of generation of $\psi$:

\begin{lem}
$\ker(\psi)$ is generated in less than
\[
	\tau = \sigma
	+ \max \left( \max_{i\in \Te}(\ell_i \gcd(a_i, b_i)), \max_{\substack{
	i \in	\Tp \\ j\in \Tm}} (\ell_{i,j} (\bida_i \bidb_j - \bida_j \bidb_i))
	\right).
\]
\end{lem}

\begin{proof}
We first claim that $\psi$ there are $\beta_1, \ldots, \beta_n$ \todo{Aaron: Say where these polynomials live in}
 such that $\psi$ has relations of the form
\begin{equation}
\label{eqn:hirz-relations-psi}
	(z_i - \beta_i)\prod_{j=1}^n {z_j}^{c_{j}}
\end{equation}

\noindent
for all $d \in \mathbb{N}$ \todo{Aaron: This confuses me because $d$ is not used above, is $d$ supposed to be $i$?}
 and $c_i \ge -\alpha_i d$ satisfying $\bida_i + \sum_{j = 1}
^n \bida_j c_j = 0$ and $\bidb_i + \sum_{i=1}^n \bidb_j c_j = 0$.

To find such a $\beta$, \todo{Aaron: why is there no $i$ subscript? I changed this in the $\bp^m$ case, probably copy past, mutatis mutandis}
 we simply write $f_i$ as a 
polynomial in $f_1, \ldots f_4$. Furthermore, these
generate all relations, since they allow us to reduce any $\prod_{j =
1}^n f_j^{c_j}$ to the canonical form where $r_i = - \lfloor d \alpha_i \rfloor$ whenever $i > 4$. \todo{Add $\ell_i$ defenition}

%Note that we can view a relation from Equation
%~\ref{eqn:hirz-relations-psi} as an element of $S$ given
%by $(d, c_1, \ldots, c_{i-1}, c_i + 1, c_{i+1}, \ldots , c_n)$.  
Let $\Sigma'\subseteq \Sigma$ be the \todo{change all $S$ to sigma's}
set
\begin{align*}
	\Sigma' = & \{(d, c_1, \ldots, c_{i-1}, c_i + 1, c_{i+1}, \ldots , c_n)  \in \bz^{n+1}_{\geq 0} : \\
	& c_i \ge -\alpha_i d, \text{ for all i, and } \bida_i + \sum_{j = 1}^m \bida_j c_j = \bidb_i + \sum_{j=1}^n \bidb_j c_j
= 0 \}
\end{align*}

Let $\sigma\in \Sigma'$ correspond to the monomial $y_i \prod_{j=1}^n {y_j}^{c_j}$.  Then we can write $\sigma$ as a sum of $r_j e_j$'s for $j\in \Te$ and $r_{j,k}e_{j,k}$ for $j\in \Tp, k\in \Tm$.  
For convenience, let $e_j$ (when it exists) be in degree $d_j$ with $j^{th}$ component $-\alpha_j d_j + \kappa_j$ and let $e_{i,j}$ (when it exists) lie in degrees $d_{i,j}$ with $i^{th}$ component $\kappa_{i,j}'$ and $j^{th}$ component $\kappa_{j,i}''$.
Then a careful computation deduces that 
%if $i\in \Te$, $i\in \Tp$, and $i\in \Tm$, then
\[
	\sigma = \lambda + E
\]
where $\lambda \in \Sigma$ and
\begin{equation}\label{eqn:hirz-E-translation}
	E = \begin{cases}
	\frac{1}{\kappa_i} e_i  	&  \mbox{ if }i\in \Te\\
	\sum_{j\in \Tm} r_{i,j} e_{i,j}: r_{i,j}\in \br, \sum_{j\in \Tm} r_{i,j}\kappa_{i,j}' = 1 & \mbox{ if } i\in \Tp\\
	\sum_{j\in \Tp} r_{j,i} e_{j,i}: r_{j,i} \in \br, \sum_{j\in \Tm} r_{j,i} \kappa_{i,j}'' = 1 	& \mbox{ if } i\in \Tm\\
	\end{cases}.
\end{equation}

\noindent
From this, we can deduce that
\[
	\deg(E) \le \begin{cases}
	\ell_i \gcd(\bida_i, \bidb_i)	&\mbox{ if } i \in \Te \\
	\max_{j \in \Tm} \bigl(\ell_{i, j} (\bida_i \bidb_j - \bida_j \bidb_i)\bigr)
	&\mbox{ if } i \in \Tp \\
	\max_{j \in \Tp} \bigl(\ell_{j, i} (\bida_j \bidb_ i - \bida_i \bidb_j) \bigr)
	&\mbox{ if } i \in \Tm \end{cases}.
\]
Let $M$ be the set of monomials terms of $\beta_j$.  Then for $\mu\in M$, we can further compute that the lattice point corresponding to $\mu \prod_{j=1}^n {f_i}^{c_i}$ is of the form
\[
	\sigma + E_\mu
\]
where $E_\mu$ is defined similarly to $E$ (in particular, they lie in the same degree).

Let $F\in R_D$ corresponding to the lattice point $E$ and let $F_\mu$ correspond to the lattice point $E_\mu$.  
Then $F - \sum_{\mu \in M} F_\mu = 0$ induces a
relation in degree at most
\[
	\sigma + \deg(F)
\]

\noindent
which divides $(z_i - \beta_i) \prod_{i = 1}^n y_i^{c_i}$.
We can bound $\deg(F)$ (which equals $\deg(F_\mu)$ for all $\mu$) by
\[
	\deg(F) = \deg(E) \le \begin{cases}
	\ell_i \gcd(\bida_i, \bidb_i)	&\mbox{ if } i \in \Te \\
	\max_{j \in \Tm} \bigl(\ell_{i, j} (\bida_i \bidb_j - \bida_j \bidb_i)\bigr)
	&\mbox{ if } i \in \Tp \\
	\max_{j \in \Tp} \bigl(\ell_{j, i} (\bida_j \bidb_ i - \bida_i \bidb_j) \bigr)
	&\mbox{ if } i \in \Tm \end{cases}.
\]

\noindent
Therefore, $\ker(\psi)$ is generated in degree at most
\[
	\tau = \sigma
	+ \max \left(\max_{i\in \Te} \bigl(\ell_i \gcd(a_i, b_i) \bigr),
	\; \max_{\substack{i \in \Tp \\ j \in \Tm}} \bigl(\ell_{i, j}
	(\bida_i \bidb_j - \bida_j
	\bidb_i) \bigr) \right).
\]
\end{proof}

Combining these results, we get the following theorem:

\begin{thm}
\label{thm:hirz-generators-relations}
Let $D = \sum_{i=1}^n \alpha_i D_i \in \bq \otimes \di \hirz_k.$ Let $D_i = V(f_i)$ where $f_i \in \sco(a_i, b_i)$; further suppose $f_1$ and $f_2$ are independent linear
polynomials in $x_0$ and $x_1$ and $f_3$ and $f_4$ are independent
linear polynomials in $y_0$ and $y_1$.
Then $R_D$ is generated in degrees at most
\[
	\sigma = \sum_{i\in \Te(D)} \gcd(\bida_i, \bidb_i)\ell_i +
	\sum_{\substack{i \in \Tp(D) \\	j \in \Tm(D)}} (\bida_i \bidb_j
	- \bida_j \bidb_i) \ell_{i, j}
\]

\noindent
with relations generated in degrees at most 
\[
	\max(2 \sigma, \sigma + \tau)
\]

\noindent
where
\[
	\tau = \sigma
	+ \max \left( \max_{i\in \Te} \bigl(\ell_i \gcd(a_i, b_i) \bigr),
	\max_{\substack{i \in \Tp \\ j \in \Tm}} \bigl(\ell_{i, j}
	(\bida_i \bidb_j - \bida_j \bidb_i) \bigr) \right).
\]
\end{thm}
\todo{Aaron: Shouldn't $\tau$ not include $\sigma$?}


\todo{Aaron: I know exactly what is going on, but please spell out the proof in 2 lines for completeness.}

%%%%%%%%%%%%%%%%%%%%%%%%% Further Questions %%%%%%%%%%%%%%%%%%%%%%%%%%%%

\section{Conclusion}
\label{sec:conc}
This paper extends the work of O'Dorney \cite{dorney:canonical} on $\bp^1$ to $\bp^m$ and Hirzebruch Surfaces, bounding the degrees of generation and relations of canonical rings of arbitrary divisors $D$.  The methods for deducing bounds on generators and relations of $\hirz_k$ applied similar methods to that of O'Dorney's for $\bp^1$.  

We can view $\hirz_0$ as $\bp^1\times \bp^1$ and more generally $\hirz_k$ as a $\bp^1$ bundle over $\bp^1$.  One might hope that similar ideas could be generalized to all bundles.  Specifically:
\begin{question}
\label{qn:general-product-bundle}
Suppose $X$ and $Y$ are schemes.  Let $D$ be a divisor on $X\times Y$ or more generally on an $X$ bundles over $Y$.  Can we describe $D$ in terms of divisors $D_1, \ldots D_n$ on $X$ and $D_1', \ldots d_n'$ on $Y$.  Furthermore, can we find generators for $R_D$ and its ideal of relations using those of the $R_{D_i}$'s and $R_{D_j'}$'s?
\end{question}

An immediate obstruction to this is the fact that we critically used the fact that $\hirz_k$ has well-defined bi-degrees.  Unfortunately, in general $\Pic(X \times Y) \not \cong \Pic(X) \times \Pic(Y)$.

Alternatively, we can view $\hirz_1$ as $\bp^2$ blown up at a point.  This suggests the following question:
\begin{question}
\label{qn:general-blowup}
Let $X$ be a scheme.  Let $X'$ be the blowup of $X$ at a point $P$ with a divisor $D$.  First, is there a way to express $D$ in terms of divisors $D_1, \ldots, D_n$ on $X$?  Furthermore, is there a way to describe generators and relations of $R_D$ in terms of those of $R_{D_i}$ for all $i$.
\end{question}

Fortunately, the Picard group of $X$ blown up at a point is the same as that of $X$ times an additional copy of $\bz$.  Perhaps analyzing the argument from Lemma ~\ref{lem:hirz-generators-relations} \todo{make sure this is the lemma and not the theorem with the same name} in the case of $\hirz_k$ from the perspective of adding a copy of $\mathbb{Z}$ to the Picard group, as induced by the blowup, might suggest a general method for blowups.

Answering Question ~\ref{qn:general-blowup} would in fact provide an answer for all rational surfaces, via the minimal model program for surfaces \todo{Reference}.  One might to provide an analogous result to Lemma ~\ref{lem:hirz-generators-relations} \todo{make sure this is the lemma and not the theorem with the same name} in the case of all minimal surfaces:
\begin{question}
\label{qn:general-minimal-surface}
Can we describe generators and relations of $R_D$ for $D$ a divisor on an arbitrary minimal surface $X$?
\end{question}

%%%%%%%%%%%%%%%%%%%%%%%%% Acknowledgements %%%%%%%%%%%%%%%%%%%%%%%%%%%%

\section{Acknowledgments}
\label{sec:ack}
We are grateful to David Zureick-Brown for introducing us to this
subject, for providing invaluable guidance,
and for his mentorship. We also thank Ken Ono and the
Emory University Number Theory REU for arranging our project and
providing a great environment for mathematical learning and
collaboration.
Finally, we gratefully acknowledge the financial support given by
NSF Grant Award Number 1250467 via the Emory University Number
Theory REU. We deeply appreciate all of the support that has made
our work possible.

%%%%%%%%%%%%%%%%%%%%%%%%%%%% References %%%%%%%%%%%%%%%%%%%%%%%%%%%%%%%

\nocite{*}
\bibliography{bibliography-stacky-surface}{}
\bibliographystyle{plain}

\end{document}