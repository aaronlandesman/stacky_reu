%%%%%%%%%%%%%%%%%%%%%
%   AMS packages    %
%%%%%%%%%%%%%%%%%%%%%
\documentclass{amsart}

\usepackage{amsmath}
\usepackage{amsxtra}
\usepackage{amscd}
\usepackage{amsthm}
\usepackage{amsfonts}
\usepackage{amssymb}
\usepackage{eucal}
\usepackage[all]{xy}
\usepackage{graphicx}
\usepackage{tikz-cd}
\usepackage{mathrsfs}
\usepackage{subfiles}
%\usepackage{mathpazo} not a huge fan
\usepackage{euler}
\usepackage{hyperref}
\usepackage{color}
\usepackage{longtable}
\usepackage{float}
\usepackage{caption}

\usepackage[colorinlistoftodos, textsize=tiny]{todonotes}
\def\listtodoname{List of Todos}
\def\listoftodos{\@starttoc{tdo}\listtodoname}

%\addtolength{\oddsidemargin}{-.5 in}
	%\addtolength{\evensidemargin}{-.4 in}
	%\addtolength{\textwidth}{1 in}
%
	%\addtolength{\topmargin}{0 in}
	%\addtolength{\textheight}{0 in}

\RequirePackage{color}
\definecolor{myred}{rgb}{0.75,0,0}
\definecolor{mygreen}{rgb}{0,0.5,0}
\definecolor{myblue}{rgb}{0,0,0.65}

%\usepackage{hyperref}
  %\hypersetup{colorlinks=true,citecolor=blue}

\usepackage{tikz}
\usepackage{tikz-cd}
\usetikzlibrary{matrix,arrows,decorations.pathmorphing}

%%%%%%%%%%%%%%%%%%%%%%% amsthm theorem styles %%%%%%%%
%%%%%%%%%%%%%%%

\theoremstyle{plain}
  \newtheorem{thm}{Theorem}[section]
  \newtheorem{prop}[thm]{Proposition}
  \newtheorem{lem}[thm]{Lemma}
  \newtheorem{cor}[thm]{Corollary}
	\newtheorem{claim}[thm]{Claim}
	\newtheorem{question}[thm]{Question}
	
\theoremstyle{definition}
  \newtheorem{defn}[thm]{Definition}
  \newtheorem{example}[thm]{Example}
  \newtheorem{exer}[thm]{Exercise}
  \newtheorem{ctexample}[thm]{Counterexample}
  \newtheorem{convention}[thm]{Convention}
	\newtheorem{conjecture}[thm]{Conjecture}
	
\theoremstyle{remark}
	\newtheorem{rem}[thm]{Remark}
  \newtheorem{note}[thm]{Notation}
  \newtheorem*{note*}{Notation}
  \newtheorem{case}{Case}
	
\numberwithin{equation}{section}

%%%%%%%%%%%%%%%%%%%%%%%%% custom commands %%%%%%%%%%%%%%%%%%%%%%%%%%%

\newcommand\nc{\newcommand}
\nc\on{\operatorname}
\nc\renc{\renewcommand}
\newcommand\ssec{\subsection}
\newcommand\sssec{\subsubsection}
\newcommand\bh{{\mathbb H}}
\newcommand\bn{{\mathbb N}}
\newcommand\bc{{\mathbb C}}
\newcommand\bbf{{\mathbb F}}
\newcommand\br{{\mathbb R}}
\newcommand\bq{{\mathbb Q}}
\newcommand\bp{{\mathbb P}}
\newcommand\bz{{\mathbb Z}}
\newcommand\ba{{\mathbb A}}

\newcommand\sco{{\mathscr O}}

\newcommand{\id}{\mathrm{id}}
\newcommand\im{\text{im }}
\newcommand\coker{\text{coker}}
\newcommand\gal{\mathrm{Gal}}

\DeclareMathOperator{\ord}{ord}
\DeclareMathOperator{\sym}{Sym}

%%%%%%%%%%%%%%%%%%%%% cring custom commands %%%%%%%%%%%%%%%%%%%%%%%%%

\DeclareMathOperator\di{Div}
\newcommand\sx{\mathscr X}
\newcommand \subhalf[1]{\frac{{#1} - 1}{2{#1}}}
\newcommand{\se}[1]{\section*{Problem #1}}
\newcommand{\halfcan}{L}
\DeclareMathOperator{\supp}{Supp}
\DeclareMathOperator{\initial}{in_\prec}
\DeclareMathOperator{\gin}{gin}
\DeclareMathOperator{\Eff}{Eff}
\DeclareMathOperator{\sat}{sat}
\DeclareMathOperator{\newspan}{span}
\DeclareMathOperator{\proj}{Proj}
\DeclareMathOperator{\spec}{Spec}
%\captionsetup[table]{belowskip = 4pt}

\makeatletter
\newcommand{\customlabel}[2]{%
   \protected@write \@auxout {}{\string \newlabel {#1}{{#2}{\thepage}{#2}{#1}{}} }%
   \hypertarget{#1}{#2}
}
\makeatother

%%%%%%%%%%%%%%%%%%%%%%%%%%%%%% title %%%%%%%%%%%%%%%%%%%%%%%%%%%%%%%%

\title{Spin canonical rings of log stacky curves}

\author{Aaron Landesman}
\address[Aaron Landesman]{Department of Mathematics, Harvard University}
\email{aaronlandesman@college.harvard.edu}

\author{Peter Ruhm}
\address[Peter Ruhm]{Department of Mathematics, Stanford University}
\email{pruhm@stanford.edu}

\author{Robin Zhang}
\address[Robin Zhang]{Department of Mathematics, Stanford University}
\email{robinz16@stanford.edu}

\date{\today}

%%%%%%%%%%%%%%%%%%%%%%%%%%%%% document %%%%%%%%%%%%%%%%%%%%%%%%%%%%%%

\begin{document}

\begin{abstract}
 	Bonding generator and relation degrees of general $\bq$-divisors
	of surfaces.
\end{abstract}

\maketitle

%%%%%%%%%%%%%%%%%%%%%%%%%%%% Introduction %%%%%%%%%%%%%%%%%%%%%%%%%%%%%%%

\section{Introduction}

\section{Background}

\ssec{Notation}
\begin{convention}
Write
\begin{align*}
	D = \sum_{i=1}^{n}\alpha_i D_i.
\end{align*}
where $\alpha_i \in \bq$.
\end{convention}

\begin{convention}
Let $\{x_0, \ldots, x_m\}$ be coordinates for $\bp^m$. Let
$\vec{v} = (v_0, \ldots, v_m) \in \br^{m + 1}$. Then write
\[
	x_{\vec{v}} := \prod_{i = 0}^{m} x_i^{v_i}
\]
\end{convention}

\section{Lemmas for Generation and Relations}

\begin{prop}
\label{prop:cone-generation}
Let $n,t \in \bz,$ let $\alpha_1, \ldots, \alpha_n \in \bq$ and let $k_i^j \in \bz$ with $1 \leq i \leq n, 1 \leq j \leq t.$ Define
\begin{align*}
	\Sigma = \{(d,c_1, \ldots, c_n) \in \bz^{n+1} : c_i \geq - d \alpha_i,1 \leq i \leq n \text{ and } \sum_{i=1}^{n}k_i^j = 0, 1 \leq j \leq t\}
\end{align*}
Let $e_1, \ldots, e_r \in \Sigma,$ with $e_i = (\delta^i, c_1^i, \ldots, c_n^i)$ be a set of {\bf extremal rays} of $\Sigma$, meaning that the $\Sigma$ is contained in the $R_{\geq 0}$ span of $e_1, \ldots, e_n$.  Then, as a semigroup, $\Sigma$ is generated by elements whose first coordinate is less than $\sum_{i=1}^{n}\delta_i$.
\end{prop}
\begin{proof}
By assumption, $\sigma \in \Sigma$ as $\sigma = \sum_{i=1}^{n}a_i e_i$ with $a_i \in \br$. Let $\{a\} := a - \lfloor a \rfloor$ denote the fractional part of $a$. Let $\lambda = \sum_{i=1}^{n}\{a_i\}$. Hence, we can write $\sigma = \lambda + \sum_{i=1}^{n}\lfloor a_i \rfloor e_i.$ Hence, $\sigma$ lies in the $\bz_{\geq 0}$ span of $\lambda, e_1,\ldots, e_n$, which all have first coordinate less than $\sum_{i=1}^{n}\delta_i$. Therefore, $\Sigma$ is generated by elements whose first coordinate is less than $\sum_{i=1}^{n}\delta_i$.
\end{proof}




\section{Canonical Rings on Projective Space}
In this section, we bound the degrees of generators and relations for divisors on $\bp^m$, for all $m \geq 1$. The $n = 1$ case was covered in \cite{dorney:canonical}.

\ssec{Preliminaries on Projective Space}

For the remainder of this section, we shall fix $m \geq 1$ and choose an isomorphism $\bp^m \cong \proj V$ so that $x_0,\ldots, x_m$ form a basis for $V$. Through the rest of this section, we will think of $x_i$ as a rational section of $H^0(\bp^m, \sco_{\bp^m})$ of degree 1.


Write
\begin{align*}
	D = \sum_{i=0}^{n}\alpha_i D_i.
\end{align*}
where $\alpha_i \in \bq$ and $\deg D_i = a_i$. Let $f_i$ be Cartier divisors such that $D_i = V(f_i)$. We shall further assume that among there exist functions $f_0,\ldots, f_{m+1}$ form a basis for degree one rational functions on $\bp^m$.

\begin{prop}
\label{prop:pm-span-and-basis}
The functions $u^d \cdot \prod_{i=1}^n f_i^{c_i}$ so that 
\begin{align}
\label{align:pm-span}
\sum_{i=0}^{n} c_i \cdot a_i \cdot f_i = 0 & \text{ and } &c_i \geq \lfloor \alpha_i d\rfloor	
\end{align}
form a spanning set for $H^0(\bp^m, dD)$ over $k$. Furthermore, functions 
of the form $u^d \cdot \prod_{i=1}^n f_i^{c_i} $ such that
\begin{align}
\label{align:pm-basis}
c_i = -\lfloor d\alpha_i \rfloor & \text{ for } & i > m
\end{align}
form a basis for $H^0(\bp^m, dD)$ over $k$.
\end{prop}
\begin{proof}
By definition of $H^0(\bp^m,dD)$, functions satisfying conditions 
~\eqref{align:pm-span} lie in $H^0(\bp^m,dD)$. To complete the proof, it suffices to check functions satisfying conditions ~\eqref{align:pm-basis} form a basis of $H^0(\bp^m,dD)$. Note that there $\binom{m+ \lfloor dD \rfloor }{m}$ functions satisfying condition ~\eqref{align:pm-basis}. However we know $h^0(\bp^m,dD) = \binom{m+ \lfloor dD \rfloor }{m},$ so it suffices to show that those functions satisfying condition ~\eqref{align:pm-basis} are independent. This follows from the assumption that $f_0,\ldots, f_n$ form a basis of degree 1 rational functions, and so degree $\lfloor d \deg D \rfloor $ monomials in $f_0,\ldots, f_n$ form a basis of degree $\lfloor d \deg D \rfloor $ rational functions.
\end{proof}

\begin{rem}
Remark about how we can assume divisor only consists of hyperplanes
by taking the Veronese embedding.
\todo{do this}
\end{rem}

\ssec{One point on Projective Space}
\label{ssec:proj-one-point}

\begin{thm}
\label{thm:proj-one-point}
Let $X = \bp^m$ with $\{x_0, \ldots, x_m\}$ a coordinate system for
$\bp^m$. Let $D = \frac{a}{b} H$, with $a, b \in \bz$ and $H =
V(x_k)$ a hyperplane of $\bp^m$, be a divisor on $\bp^m$. Let
\[
	\left\lfloor \frac{a}{b} \right\rfloor = \frac{c_0}{d_0} < \frac{c_1}{d_1} <
	\ldots < \frac{c_r}{d_r} = \frac{a}{b}
\]

\noindent
be the best lower approximations of $\frac{a}{b}$. Then the
canonical ring
\[
	R_D := \bigoplus_{d \geq 0} H^0(\bp^m, \lfloor dD \rfloor)
\]

\noindent
has a minimal presentation consisting of the $\sum_{i = 0}^{r}
{{m + i} \choose {i}}$ generators $f_{\vec{v}} := \frac{u^{a_i}
x^{\vec{v}}}{x_k^{c_i}}$ where $\vec{v} \in \bz_{\geq 0}^{m + 1}$
such that $\sum_{j = 0}^{m} v_j = c_i$ and $z$ \todo{how many?}
relations of the form
\[
	x^{\vec{u}} x^{\vec{v}} = x^{\vec{w}} x^{\vec{z}}
\]

\noindent
when $\vec{u} + \vec{v} = \vec{w} + \vec{z}$.
\end{thm}

\begin{proof}
We know $f_0$ is a minimal generator in degree $1$. If $x^{\vec{v}}$
were not minimal for $\deg \vec{v} = c_i > 0$ for some $i > 0$, then
$f_{\vec{v}} = f_{\vec{w}} f_{\vec{z}}$ for some $f_{\vec{w}},
f_{\vec{z}} \in R_D$ with $\deg \vec{w}, \deg \vec{z} < \deg \vec{v}$.
\end{proof}



\ssec{Bounds for Arbitrary Divisors on Projective Space}

\begin{rem}
When $\langle f_1, \ldots, f_r \rangle = \langle x_0, \ldots, x_n \rangle$ (the irrelevant ideal), we may directly use the hypersurface method by considering $x_0, \ldots, x_n$ instead of the $f_i$'s. However, when $\sqrt{\langle f_1, \ldots, f_r \rangle} = \langle x_0, \ldots, x_n \rangle$ (or the saturation), we cannot obtain the same result from our modification of O'Dorney's method as seen in Example ~\ref{eg:radical}. Instead we must use ghost points and hypersurfaces.
\end{rem}

\begin{example}
\label{eg:radical}
As an example for why we need to add in ghost points for $\bp^2$, consider $\frac{-1}{5}V(x_0^2) + \frac{1}{7}V(x_1^2) + \frac{1}{17}V(x_2^2)$. In degree $595 = 5* 7 * 17$, this has dimension $6$, but we cannot possibly use the lattice approach, as all lower degrees only have dimension 1. In fact, we may construct similar examples using any $x_i^m$ for $f_i$ with appropriate coefficients.
\end{example}
\todo{construct class of additional examples}
\todo{The example above needs to be changed because $x_0^2$ can be replaced by $2x_0$}

\section{Canonical Rings on \texorpdfstring{$F_k$}{}}
Let $D=\sum_{i=1}^m \alpha_i C_i$ for some curve $C_i\in F_k$, such that $C_1, \ldots, C_4$ are independent curves with bi-degrees (1,0), (1,0), (0,1), and (0,1) respectively.  Let $C_i = V(f_i)$ where $f_i \in \mathscr{O}(a_i, b_i)$.  Then each $C_i$ corresponds to a polynomial $f_i$ where $f_1$ and $f_2$ are independent linear polynomials in $x_0$ and $x_1$ and $f_3$ and $f_4$ are independent linear polynomials in $y_0$ and $y_1$.
\begin{defn}
Suppose $D = \sum_{i=1}^m \alpha_i C_i \in \mathbb{Q} \otimes \di(F_k)$.  Then define 
\begin{equation}\label{eqn:define-T=(D)}
	T_=(D) = \left\{i \in \{1, \ldots, m\}: a_i \sum_{k=1}^m b_k \alpha_k = b_i \sum_{k=1}^m a_k \alpha_k \right\},
\end{equation}
\begin{equation}\label{eqn:define-T+(D)}
	T_+(D) = \left\{ i \in \{1, \ldots, m\}:  \vphantom{\sum_{k=1}^m} a_i \sum_{k=1}^m \alpha_k b_k > b_i \sum_{k=1}^m \alpha_k a_k \right\},
\end{equation}
and
\begin{equation}\label{eqn:define-T-(D)}
	T_-(D) = \left\{ j \in \{1, \ldots, m\}: a_j \sum_{k=1}^m \alpha_k b_k < b_j \sum_{k=1}^m \alpha_k a_k s \right\}.
\end{equation}
\end{defn}

\begin{lem}
$R_D$ is generated in degree at most
\[
	\sum_{i\in T_=(D)} \gcd(a_i, b_i)\ell_i + \sum_{i\in T_+(D) \atop j\in T_-(D)} (a_i b_j- a_j b_i)\ell_{i,j}.
\]
\end{lem}
\begin{proof}
Let $(a_i, b_i)$ be the bi-degrees of $f_i$. 

Notice that if $\gamma \in (R_D)_d$, then 
\[
	g = \sum_{i=1}^m {f_i}^{c_i}
\]
where each $c_i \ge - \alpha_i c_i$, $\sum_{i=1}^m c_i a_i = 0$, and $\sum_{i=1}^m c_i b_i = 0$.  Then we can view $g$ inside of the lattice
$S\subseteq \mathbb{R}^{n+1}$ given by ordered elements of the form $(d, c_1, \ldots, c_n)$ satisfying $c_i \ge - d \alpha_i$ for all $i$, $\sum_{i=1}^m c_i a_i = 0$, and $\sum_{i=1}^m c_i b_i = 0$. 
From here, determining a spanning set for $(R_D)_d$ reduces to finding the extremal rays of $S$ \todo{cite previous lemma}.  

To do this, we extend the method of O'Dorney \todo{cite}.  We first view the cone $T\subseteq \mathbb{R}^{n+1}$ given by elements of the form $(d, c_1, \ldots, c_n)$ satisfying $c_i \ge -\alpha_i d$ for all $i$ and $\sum_{i=1}^n c_i (a_i + b_i) =0$.  By O'Dorney, this has extremal rays given by 
\[
	e_i = (1, -\alpha_1, \ldots, -\alpha_{i-1}, \frac{\sum_{j \ne i} \alpha_j(a_j+ b_j)}{a_i + b_i}, -\alpha_{i+1}, \ldots, - \alpha_m).
\]
Let $\Delta^{n-1}$ be the simplex with vertices $e_1, \ldots e_m$.  We can then intersect the hyperplane $H$ given $\sum_{i=1}^m  a_i X_i = 0$.  The extremal rays of $S$ are given by the set of $e_i$ contained in $H$ together with the points of intersection $e_{i,j}$ of $H$ and the line segment connecting $e_i, e_j \not \in H$.

 We can calculate the following elements of $R_D$ whose corresponding lattice points lie on these extremal rays:

For $i \in T_=(D)$
define
\[
	\epsilon_i := (\prod_{k \ne i} f_k^{-\alpha \ell_i \gcd(a_i, b_i)}) (f_i^{\ell_i \frac{\gcd(a_i, b_i)}{a_i + b_i}\sum_{k \ne i} \alpha_k (a_k + b_k)}) \in H^0(\ell_i \gcd(a_i, b_i) D).
\]
For $i \in T_+(D)$ and $j \in T_-(D)$, define
\[
	\epsilon_{i, j} = (\prod_{k = 1}^m f_k^{-\alpha \ell_{i,j} (a_i b_j - a_j b_i)}) {f_i}^{s_1} {f_j}^{s_2} \in H^0(\ell_{i,j}(a_i b_j - a_j b_i)D)
\]
where
\[
	s_1 = b_j \sum_{k \ne i,j} \alpha_k a_k - a_j \sum_{k\ne i, j} \alpha_k b_k
\]
and
\[
	s_2 = b_i \sum_{k \ne i,j} \alpha_k a_k - a_i \sum_{k \ne i, j} \alpha_k b_k.
\]
Let $E$ be the set of $\epsilon_i$ and $\epsilon_{i,j}$ given above, satisfying the respective conditions.
$E$ satisfies the conditions of proposition \todo{cite earlier proposition}, so by applying that proposition $R_D$ is generated in degree at most
\[
	\sum_{i\in T_=(D)} \gcd(a_i, b_i)\ell_i + \sum_{i\in T_+(D) \atop j\in T_-(D)} (a_i b_j- a_j b_i)\ell_{i,j}.
\]
\end{proof}

\todo{factor $\phi$ through $\psi$ and $\chi$.}
\begin{lem}
$\ker(\psi)$ is generated in degree at most \todo{insert number}.
\end{lem}
\begin{proof}
We first claim that $\psi$ has a relation of the form
\[
	(y_j - \beta_j)\prod_{i=1}^m {y_i}^{c_{i}}
\]
whenever $d\in \mathbb{N}$, $c_i \ge -\alpha_i d$, and $\beta_j \in k[y_1, \ldots, y_4]$ satisfies $\psi(\beta_j)\in \mathscr{O}(a_j, b_j)$.
To find such a $\beta$, we simply write $\epsilon_j$ as a polynomial in $\epsilon_1, \ldots \epsilon _4$.  


\end{proof}


%%%%%%%%%%%%%%%%%%%%%%%%% Acknowledgements %%%%%%%%%%%%%%%%%%%%%%%%%%%%

\section{Acknowledgments}

We are grateful to David Zureick-Brown for introducing us to this
subject, for providing invaluable guidance,
and for his mentorship. We also thank Ken Ono and the
Emory University Number Theory REU for arranging our project and
providing a great environment for mathematical learning and
collaboration.
Finally, we gratefully acknowledge the financial support given by
NSF Grant Award Number 1250467 via the Emory University Number
Theory REU. We deeply appreciate all of the support that has made
our work possible.

%%%%%%%%%%%%%%%%%%%%%%%%%%%% References %%%%%%%%%%%%%%%%%%%%%%%%%%%%%%%

\nocite{*}
\bibliography{bibliography-stacky-surface}{}
\bibliographystyle{plain}

\end{document}