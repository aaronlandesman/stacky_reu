%        File: changes-summary.tex
%     Created: Sat Oct 17 11:00 AM 2015 E
% Last Change: Sat Oct 17 11:00 AM 2015 E
%
%%%%%%%%%%%%%%%%%%%%%
%   AMS packages    %
%%%%%%%%%%%%%%%%%%%%%
\documentclass[10 pt]{amsart}

\usepackage{amsmath}
\usepackage{amsxtra}
\usepackage{amscd}
\usepackage{amsthm}
\usepackage{amsfonts}
\usepackage{amssymb}
\usepackage{eucal}
\usepackage[all]{xy}
\usepackage{graphicx}
\usepackage{tikz-cd}
\usepackage{mathrsfs}
\usepackage{subfiles}
\usepackage{mathpazo}
\usepackage{euler}
\usepackage[colorinlistoftodos, textsize=tiny]{todonotes}

\addtolength{\oddsidemargin}{-.5 in}
\addtolength{\evensidemargin}{-.4 in}
\addtolength{\textwidth}{1 in}
\addtolength{\topmargin}{0 in}
\addtolength{\textheight}{0 in}

\RequirePackage{color}
\definecolor{myred}{rgb}{0.75,0,0}
\definecolor{mygreen}{rgb}{0,0.5,0}
\definecolor{myblue}{rgb}{0,0,0.65}

\usepackage{hyperref}
\hypersetup{colorlinks=true,citecolor=blue}
\usepackage{tikz}
\usetikzlibrary{matrix,arrows,decorations.pathmorphing}

\theoremstyle{plain}
\newtheorem{theorem}{Theorem}[section]
\newtheorem{proposition}[theorem]{Proposition}
\newtheorem{lemma}[theorem]{Lemma}
\newtheorem{corollary}[theorem]{Corollary}
\theoremstyle{definition}
\newtheorem{definition}[theorem]{Definition}
\newtheorem{remark}[theorem]{Remark}
\newtheorem{example}[theorem]{Example}
\newtheorem{exercise}[theorem]{Exercise}
\newtheorem{counterexample}[theorem]{Counterexample}
\newtheorem{convention}[theorem]{Convention}
\newtheorem{question}[theorem]{Question}
\newtheorem{conjecture}[theorem]{Conjecture} 
\newtheorem{warning}[theorem]{Warning}
\newtheorem{fact}[theorem]{Fact}
\theoremstyle{remark}
\newtheorem{notation}[theorem]{Notation}
\numberwithin{equation}{section}
  
\newcommand\nc{\newcommand}
\nc\on{\operatorname}
\nc\renc{\renewcommand}
\newcommand\se{\section}
\newcommand\ssec{\subsection}
\newcommand\sssec{\subsubsection}
\newcommand\bn{{\mathbb N}}
\newcommand\bc{{\mathbb C}}
\newcommand\br{{\mathbb R}}
\newcommand\bq{{\mathbb Q}}
\newcommand\bp{{\mathbb P}}
\newcommand\CF{{\mathcal F}}
\newcommand\bz{{\mathbb Z}}
\newcommand\ba{{\mathbb A}}
\newcommand\fa{{\mathfrak a}}
\newcommand\fp{{\mathfrak p}}
\newcommand\fq{{\mathfrak q}}
\newcommand\fm{{\mathfrak m}}
\newcommand\so{{\mathscr O}}
\newcommand\sg{{\mathscr G}}

\newcommand\scm{{\mathscr M}}
\newcommand\scn{{\mathscr N}}
\newcommand\scf{{\mathscr F}}
\newcommand\scg{{\mathscr G}}
\newcommand\sco{{\mathscr O}}
\newcommand\sch{{\mathscr H}}
\newcommand\scl{{\mathscr L}}
\newcommand\sci{{\mathscr I}}

\newcommand \ra{\rightarrow}
\newcommand{\id}{\mathrm{id}}
\newcommand\im{\text{im }}
\newcommand\coker{\text{coker}}
\newcommand \spec{\text{Spec }}
\newcommand \proj{\text{Proj }}
\newcommand \rspec{\textit{Spec }}
\newcommand \rproj{\textit{Proj }}
\newcommand \mg{{\mathscr M_g}}

\DeclareMathOperator\ord{ord}

\newcommand \trdeg{\text{tr. deg }}
\newcommand \codim{\text{codim}}
\newcommand \rk{\text{rk }}
\newcommand \di{\text{div }}
\newcommand \depth{\text{depth }}
\DeclareMathOperator\pic{Pic}
\DeclareMathOperator\lcm{lcm}
\DeclareMathOperator\rank{rank}
\DeclareMathOperator\vol{Vol}
\DeclareMathOperator\supp{Supp}


\def\listtodoname{List of Todos}
\def\listoftodos{\@starttoc{tdo}\listtodoname}

\title{Summary of Changes in response to Referee's Report}
\author{Aaron Landesman, Peter Ruhm, and Robin Zhang}

\begin{document}

\maketitle
\section{Changes Made}
Here are the changes we have made in response to the referee's suggestions.

\begin{enumerate}
	\item We added a remark on Gorensteinness of the log canonical ring following
	the definition of a log spin canonical ring in the background section.
	We noticed that the referee made a comment saying
	``Cohen-Macaulay and
projectively Gorenstein follows as usual (of course, provided there are
no log canonical points).'' We were a bit confused about the parenthetical, but interpreted
it to mean that the canonical ring may not be Gorenstein if it is of the form
\begin{align*}
	\oplus_n H^0(X, P + n L)
\end{align*}
for $L$ some fractional divisor and $P$ a point. Of course, in our paper, all canonical rings are of the form
\begin{align*}
	\oplus_n H^0(X, n L)
\end{align*}
for $L$ a fractional divisor, so this would not pose an issue.
	\item We included Hilbert polynomial in Examples $3.3- 3.6$.
		Throughout we reference Theorem 4.2.1 of Zhou's thesis,
		but also include remark 3.2 explaining that we could also use
		Theorem 3.1 of Buckley, Reid, and Zhou's paper.
	\item We included a Remark 2.11 on the existence of generators in degree $0 \bmod r_i$
		and $-2 \bmod r_i$ near the end of subsection 2.1.
	\item We replaced ``best lower approximation'' with Hirzebruch--Jung
		fractions. This not only allowed us to delete the unnecessary, nonstandard definition
		of ``best lower approximation,''
		but also allowed us to shorten and
		clarify our proof of the rather
		technical Lemma 4.4 (which was
		Lemma 4.6 in the previous version).
\end{enumerate}


\section{Other suggestions}
The referee also made the following notational suggestions,
and here we include reasons we would prefer not to implement them.

\begin{enumerate}
	\item One suggestion was to change the notation from $\frac{e_i - 1}{2 e_i}$
		to $\frac{s_i}{2 s_i + 1}$ where $s_i = \frac{e_i - 1}{2}$.
		First, we would like to keep our notation because we often
	list degrees of generation and relations by signatures
	of the form $(g; e_1, \ldots, e_n; \delta)$ with bounds given in
	the form $\min(5, e_1, \ldots, e_n)$. These
	seem simpler to
	consider than enumeration by signatures $(g; s_1, \ldots, s_n;
	\delta)$ with bounds given in the form $\min(5, 2s_1 + 1, \ldots,
	2s_n + 1)$. Second, because we will be using $e_i$
	so frequently in our analysis, the paper is shortened
	by writing $e_i$ instead of $2s_i+1$.
	Third, we maintained the $e_i$ notation for consistency
	with the notation established in Voight--Zureick-Brown.
	Fourth, since one of our main applications is to modular forms,
	the number theorists reading it may have an easier time translating
	to modular curves whose signature is written with $e_i$ rather than $s_i$.
	\item One suggestion was to use orbifold terminology instead
		of stacky terminology. Since this paper is in many
		ways intimately tied with Voight--Zureick-Brown,
		we would like to keep the use of stacky terminology
		in order to remain compatible with that paper.
	\item One suggestion was to remove the signature invariant
		and use the more general terminology from the
		Ice cream paper. We agree that it is
		somewhat difficult to generalize the
		signature terminology,
		and we also note that this signature is commonly used in referring
		to modular curves. Since one of the important
		applications of this paper is to rings of modular
		forms, the signature will play a useful role for
		those interested in the translation to modular curves.
		Once again, we would also like to keep the signature
		to remain compatible with Voight--Zureick-Brown.
\end{enumerate}

\end{document}


