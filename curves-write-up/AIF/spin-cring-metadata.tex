%        File: changes-summary.tex
%     Created: Sat Oct 17 11:00 AM 2015 E
% Last Change: Sat Oct 17 11:00 AM 2015 E
%
%%%%%%%%%%%%%%%%%%%%%
%   AMS packages    %
%%%%%%%%%%%%%%%%%%%%%
\documentclass[10 pt]{amsart}

\usepackage{amsmath}
\usepackage{amsxtra}
\usepackage{amscd}
\usepackage{amsthm}
\usepackage{amsfonts}
\usepackage{amssymb}
\usepackage{eucal}
\usepackage[all]{xy}
\usepackage{graphicx}
\usepackage{tikz-cd}
\usepackage{mathrsfs}
\usepackage{subfiles}
\usepackage{mathpazo}
\usepackage{euler}
\usepackage[colorinlistoftodos, textsize=tiny]{todonotes}

\addtolength{\oddsidemargin}{-.5 in}
\addtolength{\evensidemargin}{-.4 in}
\addtolength{\textwidth}{1 in}
\addtolength{\topmargin}{0 in}
\addtolength{\textheight}{0 in}

\RequirePackage{color}
\definecolor{myred}{rgb}{0.75,0,0}
\definecolor{mygreen}{rgb}{0,0.5,0}
\definecolor{myblue}{rgb}{0,0,0.65}

\usepackage{hyperref}
\hypersetup{colorlinks=true,citecolor=blue}
\usepackage{tikz}
\usetikzlibrary{matrix,arrows,decorations.pathmorphing}

\theoremstyle{plain}
\newtheorem{theorem}{Theorem}[section]
\newtheorem{proposition}[theorem]{Proposition}
\newtheorem{lemma}[theorem]{Lemma}
\newtheorem{corollary}[theorem]{Corollary}
\theoremstyle{definition}
\newtheorem{definition}[theorem]{Definition}
\newtheorem{remark}[theorem]{Remark}
\newtheorem{example}[theorem]{Example}
\newtheorem{exercise}[theorem]{Exercise}
\newtheorem{counterexample}[theorem]{Counterexample}
\newtheorem{convention}[theorem]{Convention}
\newtheorem{question}[theorem]{Question}
\newtheorem{conjecture}[theorem]{Conjecture} 
\newtheorem{warning}[theorem]{Warning}
\newtheorem{fact}[theorem]{Fact}
\theoremstyle{remark}
\newtheorem{notation}[theorem]{Notation}
\numberwithin{equation}{section}
  
\newcommand\nc{\newcommand}
\nc\on{\operatorname}
\nc\renc{\renewcommand}
\newcommand\se{\section}
\newcommand\ssec{\subsection}
\newcommand\sssec{\subsubsection}
\newcommand\bn{{\mathbb N}}
\newcommand\bc{{\mathbb C}}
\newcommand\br{{\mathbb R}}
\newcommand\bq{{\mathbb Q}}
\newcommand\bp{{\mathbb P}}
\newcommand\CF{{\mathcal F}}
\newcommand\bz{{\mathbb Z}}
\newcommand\ba{{\mathbb A}}
\newcommand\fa{{\mathfrak a}}
\newcommand\fp{{\mathfrak p}}
\newcommand\fq{{\mathfrak q}}
\newcommand\fm{{\mathfrak m}}
\newcommand\so{{\mathscr O}}
\newcommand\sg{{\mathscr G}}

\newcommand\scm{{\mathscr M}}
\newcommand\scn{{\mathscr N}}
\newcommand\scf{{\mathscr F}}
\newcommand\scg{{\mathscr G}}
\newcommand\sco{{\mathscr O}}
\newcommand\sch{{\mathscr H}}
\newcommand\scl{{\mathscr L}}
\newcommand\sci{{\mathscr I}}

\newcommand \ra{\rightarrow}
\newcommand{\id}{\mathrm{id}}
\newcommand\im{\text{im }}
\newcommand\coker{\text{coker}}
\newcommand \spec{\text{Spec }}
\newcommand \proj{\text{Proj }}
\newcommand \rspec{\textit{Spec }}
\newcommand \rproj{\textit{Proj }}
\newcommand \mg{{\mathscr M_g}}

\DeclareMathOperator\ord{ord}

\newcommand \trdeg{\text{tr. deg }}
\newcommand \codim{\text{codim}}
\newcommand \rk{\text{rk }}
\newcommand \di{\text{div }}
\newcommand \depth{\text{depth }}
\DeclareMathOperator\pic{Pic}
\DeclareMathOperator\lcm{lcm}
\DeclareMathOperator\rank{rank}
\DeclareMathOperator\vol{Vol}
\DeclareMathOperator\supp{Supp}


\def\listtodoname{List of Todos}
\def\listoftodos{\@starttoc{tdo}\listtodoname}

\title{Metadata for ``Spin canonical rings of log stacky curves''}
\author{Aaron Landesman, Peter Ruhm, and Robin Zhang}

\begin{document}

\maketitle

\begin{enumerate}
	\item Title (English): Spin canonical rings of log stacky curves \\
		Titre (Fran\c{c}ais): Anneaux log-canoniques de spin des
		courbes de champs alg\'{e}briques
	\item Abstract (English): Consider modular forms arising from a
		finite-area quotient of the upper-half plane by a Fuchsian group.
		By the classical results of Kodaira--Spencer, this ring of
		modular forms may be viewed as the log spin canonical ring of a
		stacky curve. In this paper, we tightly bound the degrees of
		minimal generators and relations of log spin canonical rings.
		As a 	consequence, we obtain a tight bound on the degrees of
		minimal generators and relations for rings of modular forms of
		arbitrary integral weight. \\
		Abstrait (Fran\c{c}ais): Consid\'{e}rez formes modulaires
		d\'{e}coulant d'une quotient d'aire-finie du demi-plan
		sup\'{e}rieur par un groupe fuchsien. Par les r\'{e}sultats
		classiques de Kodaira--Spencer, cet anneau de formes modulaires
		peut \^{e}tre consid\'{e}r\'{e} comme l'anneau log-canonique de
		spin d'un courbe de champs alg\'{e}briques. Dans cet article, nous
		lions \'{e}troitement les degr\'{e}s de g\'{e}n\'{e}rateurs
		minimaux et les relations minimes de l'anneau log-canonique de spin
		d'un courbe de champs alg\'{e}briques. Par conséquent, on obtient
		un virage serr\'{e} li\'{e} sur les degr\'{e}s de
		g\'{e}n\'{e}rateurs minimaux et relations minimes pour les anneaux
		de formes modulaires de poids arbitraires int\'{e}grantes.
	\item Running title (English): Spin canonical rings of log stacky
		curves \\
		Titre courant (Fran\c{c}ais): Anneaux log-canoniques de spin des
		courbes de champs alg\'{e}briques
	\item Keywords (English): Modular forms, canonical rings, theta
		characteristic, Petri's theorem, stacks, Groebner basis \\
		Mots-cl\'{e}s (Fran\c{c}ais): Formes modulaires, anneaux canoniques,
		th\^{e}ta-caract\'{e}ristiques, th\'{e}or\`{e}me de P\'{e}tri,
		champs alg\'{e}briques, base de Gr\"{o}bner
	\item Mathematics Subject Classification (MSC): 14Q05, 11F11
  \end{enumerate}

\end{document}


