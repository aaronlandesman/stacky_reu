%%%%%%%%%%%%%%%%%%%%%
%   AMS packages    %
%%%%%%%%%%%%%%%%%%%%%
\documentclass{amsart}

\usepackage{amsmath}
\usepackage{amsxtra}
\usepackage{amscd}
\usepackage{amsthm}
\usepackage{amsfonts}
\usepackage{amssymb}
\usepackage{eucal}
\usepackage[all]{xy}
\usepackage{graphicx}
\usepackage{tikz-cd}
\usepackage{mathrsfs}
\usepackage{subfiles}
%\usepackage{mathpazo} not a huge fan
\usepackage{euler}
\usepackage{hyperref}
\usepackage{color}
\usepackage{longtable}
\usepackage{float}
\usepackage{caption}

\usepackage[colorinlistoftodos, textsize=tiny]{todonotes}
\def\listtodoname{List of Todos}
\def\listoftodos{\@starttoc{tdo}\listtodoname}

%\addtolength{\oddsidemargin}{-.5 in}
	%\addtolength{\evensidemargin}{-.4 in}
	%\addtolength{\textwidth}{1 in}
%
	%\addtolength{\topmargin}{0 in}
	%\addtolength{\textheight}{0 in}

\RequirePackage{color}
\definecolor{myred}{rgb}{0.75,0,0}
\definecolor{mygreen}{rgb}{0,0.5,0}
\definecolor{myblue}{rgb}{0,0,0.65}

%\usepackage{hyperref}
  %\hypersetup{colorlinks=true,citecolor=blue}

\usepackage{tikz}
\usepackage{tikz-cd}
\usetikzlibrary{matrix,arrows,decorations.pathmorphing}

%%%%%%%%%%%%%%%%%%%%%%% amsthm theorem styles %%%%%%%%
%%%%%%%%%%%%%%%

\theoremstyle{plain}
  \newtheorem{thm}{Theorem}[section]
  \newtheorem{prop}[thm]{Proposition}
  \newtheorem{lem}[thm]{Lemma}
  \newtheorem{cor}[thm]{Corollary}
	\newtheorem{claim}[thm]{Claim}
	\newtheorem{question}[thm]{Question}
	
\theoremstyle{definition}
  \newtheorem{defn}[thm]{Definition}
  \newtheorem{example}[thm]{Example}
  \newtheorem{exer}[thm]{Exercise}
  \newtheorem{ctexample}[thm]{Counterexample}
  \newtheorem{convention}[thm]{Convention}
	\newtheorem{conjecture}[thm]{Conjecture}
	
\theoremstyle{remark}
	\newtheorem{rem}[thm]{Remark}
  \newtheorem{note}[thm]{Notation}
  \newtheorem*{note*}{Notation}
  \newtheorem{case}{Case}
	
\numberwithin{equation}{section}

%%%%%%%%%%%%%%%%%%%%%%%%% custom commands %%%%%%%%%%%%%%%%%%%%%%%%%%%

\newcommand\nc{\newcommand}
\nc\on{\operatorname}
\nc\renc{\renewcommand}
\newcommand\ssec{\subsection}
\newcommand\sssec{\subsubsection}
\newcommand\bh{{\mathbb H}}
\newcommand\bn{{\mathbb N}}
\newcommand\bc{{\mathbb C}}
\newcommand\bbf{{\mathbb F}}
\newcommand\br{{\mathbb R}}
\newcommand\bq{{\mathbb Q}}
\newcommand\bp{{\mathbb P}}
\newcommand\bz{{\mathbb Z}}
\newcommand\ba{{\mathbb A}}

\newcommand\sco{{\mathscr O}}

\newcommand{\id}{\mathrm{id}}
\newcommand\im{\text{im }}
\newcommand\coker{\text{coker}}
\newcommand\gal{\mathrm{Gal}}

\DeclareMathOperator{\ord}{ord}
\DeclareMathOperator{\sym}{Sym}

%%%%%%%%%%%%%%%%%%%%% cring custom commands %%%%%%%%%%%%%%%%%%%%%%%%%

\DeclareMathOperator\di{Div}
\newcommand\sx{\mathscr X}
\newcommand \subhalf[1]{\frac{{#1} - 1}{2{#1}}}
\newcommand{\se}[1]{\section*{Problem #1}}
\newcommand{\halfcan}{L}
\DeclareMathOperator{\supp}{Supp}
\DeclareMathOperator{\initial}{in_\prec}
\DeclareMathOperator{\gin}{gin}
\DeclareMathOperator{\Eff}{Eff}
\DeclareMathOperator{\sat}{sat}
\DeclareMathOperator{\newspan}{span}
\DeclareMathOperator{\proj}{Proj}
\DeclareMathOperator{\spec}{Spec}
\captionsetup[table]{belowskip = 4pt}

\makeatletter
\newcommand{\customlabel}[2]{%
   \protected@write \@auxout {}{\string \newlabel {#1}{{#2}{\thepage}{#2}{#1}{}} }%
   \hypertarget{#1}{#2}
}
\makeatother

%%%%%%%%%%%%%%%%%%%%%%%%%%%%%% title %%%%%%%%%%%%%%%%%%%%%%%%%%%%%%%%

\title{Spin canonical rings of log stacky curves}

\author{Aaron Landesman}
\address[Aaron Landesman]{Department of Mathematics, Harvard University}
\email{aaronlandesman@college.harvard.edu}

\author{Peter Ruhm}
\address[Peter Ruhm]{Department of Mathematics, Stanford University}
\email{pruhm@stanford.edu}

\author{Robin Zhang}
\address[Robin Zhang]{Department of Mathematics, Stanford University}
\email{robinz16@stanford.edu}

\date{\today}

%%%%%%%%%%%%%%%%%%%%%%%%%%%%% document %%%%%%%%%%%%%%%%%%%%%%%%%%%%%%

\begin{document}

\begin{abstract}
 	Consider modular forms arising from finite-area
	quotients of the upper-half plane by Fuchsian groups.  
	By the classical results of Kodaira--Spencer, 
	this space of modular forms may be
	viewed as log spin canonical ring of a stacky curve.
	In this paper, we tightly bound the degrees of minimal
	generators and relations of log spin canonical rings and,
	as a consequence, 
	obtain a tight bound on the degrees of minimal generators and relations  		
	for rings of modular forms of arbitrary integral weight.
\end{abstract}

\maketitle

%%%%%%%%%%%%%%%%%%%%%%%%%%%% Introduction %%%%%%%%%%%%%%%%%%%%%%%%%%%%%%%

\section{Introduction}

\section{Background}

\ssec{Notation}

Write
\begin{align*}
	D = \sum_{i=1}^{n}\alpha_i D_i.
\end{align*}
where $\alpha_i \in \bq.$


\section{Canonical Rings on Projective Space}
In this section, we bound the degrees of generators and relations for divisors on $\bp^m$, for all $m \geq 1$. The $n = 1$ case was covered in \cite{dorney:canonical}.

\ssec{Preliminaries on Projective Space}

For the remainder of this section, we shall fix $m \geq 1$ and choose an isomorphism $\bp^m \cong \proj V$ so that $x_0,\ldots, x_m$ form a basis for $V$. Through the rest of this section, we will think of $x_i$ as a rational section of $H^0(\bp^m, \sco_{\bp^m})$ of degree 1.


Write
\begin{align*}
	D = \sum_{i=0}^{n}\alpha_i D_i.
\end{align*}
where $\alpha_i \in \bq$ and $\deg D_i = d_i$. Let $f_i$ be cartier divisors such that $D_i = V(f_i)$. We shall further assume that among there exist functions $f_0,\ldots, f_{m+1}$ form a basis for degree one rational functions on $\bp^m$.

\begin{prop}
\label{prop:pm-span-and-basis}
The functions $u^d \cdot \prod_{i=1}^n f_i^{c_i}$ so that 
\begin{align}
\label{align:pm-span}
\sum_{i=0}^{n} c_i \cdot d_i \cdot f_i = 0 & \text{ and } &c_i \geq \lfloor \alpha_i d\rfloor	
\end{align}
form a spanning set for $H^0(\bp^m, dD)$ over $k$. Furthermore, functions 
of the form $u^d \cdot \prod_{i=1}^n f_i^{c_i} $ such that
\begin{align}
\label{align:pm-basis}
c_i = -\lfloor d\alpha_i \rfloor & \text{ for } & i > m
\end{align}
form a basis for $H^0(\bp^m, dD)$ over $k$.
\end{prop}
\begin{proof}
By definition of $H^0(\bp^m,dD)$, functions satisfying conditions 
~\eqref{align:pm-span} lie in $H^0(\bp^m,dD)$. To complete the proof, it suffices to check functions satisfying conditions ~\eqref{align:pm-basis} form a basis of $H^0(\bp^m,dD)$. Note that there $\binom{m+ \lfloor dD \rfloor }{m}$ functions satisfying condition ~\eqref{align:pm-basis}. However we know $h^0(\bp^m,dD) = \binom{m+ \lfloor dD \rfloor }{m},$ so it suffices to show that those functions satisfying condition ~\eqref{align:pm-basis} are independent. This follows from the assumption that $f_0,\ldots, f_n$ form a basis of degree 1 rational functions, and so degree $\lfloor d \deg D \rfloor $ monomials in $f_0,\ldots, f_n$ form a basis of degree $\lfloor d \deg D \rfloor $ rational functions.
\end{proof}

\ssec{One point on Projective Space}



\ssec{Bounds for Arbitrary Divisors on Projective Space}

%%%%%%%%%%%%%%%%%%%%%%%%% Acknowledgements %%%%%%%%%%%%%%%%%%%%%%%%%%%%

\section{Acknowledgments}

We are grateful to David Zureick-Brown for introducing us to the
study of stacky canonical rings, for providing invaluable guidance,
and for his mentorship. We also thank Ken Ono and the
Emory University Number Theory REU for arranging our project and
providing a great environment for mathematical learning and
collaboration.
Finally, we gratefully acknowledge the financial support given by
NSF Grant Award Number 1250467 via the Emory University Number
Theory REU. We deeply appreciate all of the support that has made
our work possible.

%%%%%%%%%%%%%%%%%%%%%%%%%%%% References %%%%%%%%%%%%%%%%%%%%%%%%%%%%%%%

\nocite{*}
\bibliography{bibliography-stacky-surface}{}
\bibliographystyle{plain}

\end{document}