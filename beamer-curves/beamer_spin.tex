%%%%%%%%%%%%%%%%%%%%%
%   AMS packages    %
%%%%%%%%%%%%%%%%%%%%%
\documentclass{beamer}
\usepackage{amsmath}
\usepackage{amsxtra}
\usepackage{amscd}
\usepackage{amsthm}
\usepackage{amsfonts}
\usepackage{amssymb}
\usepackage{eucal}
%\usepackage[all]{xy}
\usepackage{graphicx}
\usepackage{comment}
\usepackage{amssymb}
\usepackage{mathrsfs}
\usepackage{latexsym,amsmath,amscd,amssymb,epsfig,verbatim}
\usepackage{tikz}
\usetikzlibrary{arrows,shapes}
\usepackage{tikz-qtree}
\tikzset{level distance=50pt,
    sibling distance=7pt,
    every tree node/.style={align=center},}

\newtheorem{thm}{Theorem}
\newtheorem{cor}[thm]{Corollary}
\newtheorem{lem}[thm]{Lemma}
\newtheorem{prop}[thm]{Proposition}
\newtheorem{ex}[thm]{Exercise}
\newtheorem{conjecture}{Conjecture}
%\newtheorem*{conjecture*}{Conjecture}

\theoremstyle{remark}
\newtheorem{rem}[thm]{Remark}
\newtheorem{eg}[thm]{Example}

%\newtheorem{counterexample}[thm]{Counterexample}
\newtheorem{defn}[thm]{Definition}
%\newtheorem{claim}[thm]{Claim}
%\newtheorem{note}[thm]{Notation}
%\newtheorem{warning}[thm]{Warning}
%\newtheorem{variant}[thm]{Variant}
%\newtheorem{question}[thm]{Question}
%\newtheorem{construction}[thm]{Construction}
%\newtheorem{terminology}[thm]{Terminology}
%\newtheorem{convention}[thm]{Convention}

%%%%%%%%%%%%%%%%%%%%%%%%% custom commands %%%%%%%%%%%%%%%%%%%%%%%%%%%

\newcommand\nc{\newcommand}
\nc\on{\operatorname}
\nc\renc{\renewcommand}
\newcommand\ssec{\subsection}
\newcommand\sssec{\subsubsection}
\newcommand\BH{{\mathbb H}}
\newcommand\bO{{\mathbf O}}
\newcommand\CC{{\mathcal C}}
\newcommand\CH{{\mathcal H}}
\newcommand\BN{{\mathbb N}}
\newcommand\BC{{\mathbb C}}
\newcommand\BF{{\mathbb F}}
\newcommand\BR{{\mathbb R}}
\newcommand\BQ{{\mathbb Q}}
\newcommand\BP{{\mathbb P}}
\newcommand\BBZ{{\mathbb Z}}
\newcommand\uR{\underline{R}}
\newcommand\uZ{\underline{\BBZ}}
\newcommand\CF{{\mathcal F}}
\newcommand\uCF{\underline{{\mathcal F}}}
\newcommand\BZ{{\mathbb Z}}
\newcommand\BA{{\mathbb A}}
\newcommand\fa{{\mathfrak a}}
\newcommand\fp{{\mathfrak p}}
\newcommand\fq{{\mathfrak q}}
\newcommand\fm{{\mathfrak m}}
\newcommand\so{{\mathscr O}}
\newcommand\sg{{\mathscr G}}
\newcommand \sx{{\mathscr X}}

\newcommand\scm{{\mathscr M}}
\newcommand\scn{{\mathscr N}}
\newcommand\scf{{\mathscr F}}
\newcommand\scg{{\mathscr G}}
\newcommand\sco{{\mathscr O}}
\newcommand\sch{{\mathscr H}}
\newcommand\scl{{\mathscr L}}
\newcommand\sci{{\mathscr I}}

\newcommand{\id}{\mathrm{id}}
\newcommand\im{\text{im }}
\newcommand\coker{\text{coker}}
\newcommand \spec{\text{Spec }}
\newcommand \proj{\text{Proj }}
\newcommand \rspec{\textit{Spec }}
\newcommand \rproj{\textit{Proj }}
\newcommand{\gal}{\mathrm{Gal}}

\newcommand \trdeg{\text{tr. deg }}
\newcommand \codim{\text{codim}}
\newcommand \rk{\text{rk }}
\DeclareMathOperator\di{Div}
\newcommand \depth{\text{depth }}
\DeclareMathOperator{\ord}{ord}
\DeclareMathOperator{\sym}{Sym}

%%%%%%%%%%%%%%%%%%%%% cring custom commands %%%%%%%%%%%%%%%%%%%%%%%%%

\newcommand \subhalf[1]{\frac{{#1} - 1}{2{#1}}}
\newcommand{\se}[1]{\section*{Problem #1}}
\newcommand{\halfcan}{L}
\DeclareMathOperator{\Supp}{Supp}
\DeclareMathOperator{\initial}{in}
\DeclareMathOperator{\gin}{gin}
\DeclareMathOperator{\Eff}{Eff}
\DeclareMathOperator{\sat}{sat}

\newcommand{\GL}{\operatorname{GL}}
\newcommand{\SL}{\operatorname{SL}}
\newcommand{\PGL}{\operatorname{PGL}}
\newcommand{\PSL}{\operatorname{PSL}}


\newcommand\Wider[2][3em]{%
\makebox[\linewidth][c]{%
  \begin{minipage}{\dimexpr\textwidth+#1\relax}
  \raggedright#2
  \end{minipage}%
  }%
}


\AtBeginSection[]{
	\begin{frame}{Outline of Talk}
		\tableofcontents[currentsection]
	\end{frame}
}

%\usepackage{beamerthemeshadow}

\mode<presentation>{}
\usetheme{CambridgeUS}
\usecolortheme{beaver}

%%%%%%%%%%%%   Title slide info  %%%%%%%%%%%%%
\title{Log Spin Canonical Rings}
\author{Aaron Landesman\inst{1}, Peter Ruhm\inst{2}, Robin Zhang\inst{3}}
 
\institute[] % (optional)
{
  \inst{1}
	Harvard University,
	,
  \inst{2}
	Stanford University,
  ,
  \inst{3}
	Stanford University
}

\begin{document}

%%%%%%%%%%%%  Title Frame  %%%%%%%%%%%%%%%%
\begin{frame}
	\titlepage
\end{frame}

%%%%%%%%%%%%%%%%%%%%   Modular Forms Section   %%%%%%%%%%%%%%%%%%%
\section{Introduction: Modular Forms} 

\begin{frame}{Fuchsian Group}
%Consider rings of modular forms of Fuchsian groups.
%\pause

\begin{defn}
$\Gamma$ is a {\bf Fuchsian} group if it is a discrete subgroup of
$\PSL_2(\BR)$ acting on the upper half plane $\BH$ by fractional
linear transformations.
\end{defn}
\pause
\begin{itemize}
\item Suppose $\Gamma \backslash \BH$ has finite area. \\
\pause
\item Then $\Gamma \backslash \mathbb{H}$ can be given the structure of an {\bf{orbifold}} by completing it with cusps.  
\end{itemize}
%In presentation describe orbifold as a generalization to reimann surfaces that allow fractional points.

\end{frame}

\begin{frame}{Modular Forms}
Let $\Gamma$  be a Fuchsian group (e.g.~$\Gamma = \Gamma_0(N) \subset \SL_2(\BZ)$).

\begin{definition}
A \textbf{modular form} for $\Gamma$ of weight $k \in \BZ_{\geq 0}$ is a holomorphic function $f \colon \CH \to \BC$ such that

\begin{align*}
	&f(\gamma z) = (cz+d)^k f(z) &\text{ for all } \gamma \in \Gamma 
\end{align*}
%there was a \quad here?

\noindent
and such that the limit $\lim_{z \to *} f(z)$ exists for all cusps $*$.
\end{definition}

\begin{defn}
  Let $M_k(\Gamma)$ be the $\BC$-vector space of modular forms for $\Gamma$ of weight $k$.    
\end{defn}

\end{frame}

%%%%%%%%%%%%%%%%%%%%   Stacky Curves Section   %%%%%%%%%%%%%%%%%%%

\begin{frame}{Background: Stacky Curves}
\begin{defn}
A \textbf{stacky curve} $\sx$ over an algebraically closed field $k$
is the data of
	\begin{itemize}
		\item smooth proper connected scheme $X$ of dimension $1$ \\ 
		\item together with a finite number of closed points of $X$, $P_
			1, \ldots, P_r$, called {\bf stacky points}, with {\bf stabilizer 
			orders} $e_1, \ldots, e_r \in \BZ_{\geq 2}.$ \\ 
		\item The scheme $X$ associated to a stacky curve $\sx$ is 
			called the {\bf coarse space} of $\sx$.
	\end{itemize}
\end{defn}

\pause

Then, we describe divisors on $\sx$ by
\[
	\di \sx = \left(\bigoplus_{P\notin \{P_1, \ldots, P_r\}} \langle 
	P \rangle \right) \oplus \left(\bigoplus_{i = 1}^r \left \langle 
	\frac{1}{e_i}P_i \right \rangle \right) \subseteq \BQ \otimes \di X.
\]

\end{frame}


\begin{frame}{Background: Divisors}
We can equip stacky curves with a log divisor $\Delta$.

\begin{defn}
Divisor $\Delta$ is a \textbf{log divisor} if it is a sum of distinct
points each with trivial stabilizer.

Let $\delta := \deg \Delta$ refer to its degree.
\end{defn}

\pause
\begin{defn}
If $\sx$ has coarse space $X$ of genus $g$, then we say $\sx$ has
\textbf{signature} $\sigma = (g; e_1, \ldots, e_r; \delta)$.
\end{defn}

\pause
\begin{defn}
$\halfcan \in \di
\sx$ satisfies $2 \halfcan \sim K_X + \Delta + \sum_{i = 1}^{r}
\frac{e_i - 1}{e_i} P_i$. Such a divisor $\halfcan$ is called a
\textbf{log spin canonical divisor} on $(\sx, \Delta)$.
\end{defn}

\end{frame}


\begin{frame}{Background: Log Spin Canonical Rings}

\begin{defn}
If divisor $D \in \di \sx$ and $D = \sum_{i = 1}^{n} \alpha_i P_i$
with $\alpha_i \in \BQ$, the floor of a divisor $\lfloor D
\rfloor$ is defined to be $\lfloor D \rfloor := \sum_{i = 1}^{n}
\lfloor \alpha_i \rfloor P_i$.
\end{defn}

\pause
\begin{defn}
Let $\sx$ be a stacky curve with coarse space $X$.
If $D \in \di \sx$ is a divisor, then we define

\[
	H^0(\sx, D) := H^0(X, \lfloor D \rfloor).
\]
\end{defn}

\pause
\begin{defn}
The {\bf log spin canonical ring} of $(\sx, \Delta, \halfcan)$ is
\[
	R(\sx, \Delta, \halfcan) = \bigoplus_{d \geq 0} H^0(\sx, d \halfcan).
\]
\end{defn}

\end{frame}


\begin{frame}{Previous Work}
Let $R$ be.

\end{frame}

%%%%%%%%%%%%%%%%%%%%   Main Result Section   %%%%%%%%%%%%%%%%%%%
\section{Main Result}

\begin{frame}{Main Result}
\begin{alertblock}{Log Spin Canonical Ring Definition}
The \textbf{log spin canonical ring} of a stacky curve is
\[
	R(\sx, \Delta, \halfcan) = \bigoplus_{d \geq 0} H^0(\sx, \lfloor d \halfcan \rfloor)
\]
\end{alertblock}

Our main result gives the following bound

\pause
\begin{thm}
\label{thm:main}
Let $(\sx, \Delta, \halfcan)$ be a log spin curve over a perfect
field $k$, so that $\sx$ has signature $\sigma = (g; e_1, \ldots,
e_r; \delta)$.

Then the log spin canonical ring is generated as a $k$-algebra by 
elements of degree at most $e = \max(5, e_1, \ldots, e_r)$ with
relations generated in degrees at most $2e$,
so long as $\sigma$ does not lie in a finite list of exceptional
cases.
\end{thm}

\end{frame} 

\begin{frame}{Main Idea}
Induction on adding points.

Induction on raising stabilizer order.

\end{frame}


\begin{frame}{Adding Points}
Adding points.

\end{frame}

\begin{frame}{Raising Stabilizer Order}
Raising stabilizer order.

\end{frame}

%%%%%%%%%%%%%%%%%%%%   Inductive Section   %%%%%%%%%%%%%%%%%%%
\section{Inductive Lemmas} 

\begin{frame}{Inductive Lemmas}
Lemmas?

\end{frame}

%%%%%%%%%%%%%%%%%%%%   High Genus Section   %%%%%%%%%%%%%%%%%%%
\section{High Genus} 

\begin{frame}{High Genus}
Not sure this is necessary.

\end{frame}

%%%%%%%%%%%%%%%%%%%%   Genus One Section   %%%%%%%%%%%%%%%%%%%
\section{Genus One} 

\begin{frame}{Genus One}
This too.

\end{frame}

%%%%%%%%%%%%%%%%%%%%   Genus Zero Section   %%%%%%%%%%%%%%%%%%%
\section{Genus Zero} 

\begin{frame}{Genus Zero}
Maybe a short description.

\end{frame}

%%%%%%%%%%%%%%%%%%%%   Further Research Section   %%%%%%%%%%%%%%%%%%%
\section{Further Research} 

\begin{frame}{Further Research}
Some questions.

\end{frame}


\begin{frame}
\frametitle{Acknowledgments}
Ken Ono!
David Zureick-Brown!
Great math atmosphere!

\end{frame}

\end{document}





